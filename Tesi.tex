\documentclass[10pt]{article}
\usepackage[usenames]{color} %usato per il colore
\usepackage{amssymb} %maths
\usepackage{amsmath} %maths
\usepackage[utf8]{inputenc} %utile per scrivere direttamente in caratteri accentuati
\begin{document}
\section{Sitemap}
Una pagina web è caratterizzata da rappresentazioni multiple, come quella testuale (mediante i termini e le parole che si trovano nella pagina web), visuale (composta dall'informazione del rendering del sito) e strutturale (formata dai tag HTML). Prima di entrare nel dettaglio, bisogna definire il concetto di pagina web.

\paragraph{Sito web.} Un sito web è un grafo diretto G = (V, E), dove V è l'insieme delle pagine web ed E è l'insieme degli hyperlinks. nella maggior parte dei casi, la homepage h di un sito web rappresenta la pagina di entrata di quel sito, permettendo al sito di essere visto come un grafo diretto radicato (ovvero avente una radice).

\paragraph{Rappresentazione strutturale.} Una pagina web è caratterizzata da una rappresentazione strutturale, composta da elementi inscritti in tag HTML e organizzati secondo una struttura ad albero. Tali tag possono essere applicato a porzioni di testo, hyperlinks e dati multimediali per dare loro un significato differente e una diversa renderizzazione della pagina web.

\paragraph{Rappresentazione visuale.} Quando una pagina web viene interpretata dal browser, si dice che questa viene renderizzata dal browser: gli elementi che sono contenuti nella pagina web vengono rappresentati da scatole rettangolari, le quali possono essere affiancate oppure innestate, creando un albero chiamato Rendered Box Tree. Utilizzando un sistema di coordinate avente origine nell'angolo in alto a sinistra, tutte le posizioni degli elementi della pagina web vengono determinati. Il Rendered Box Tree, inoltre, può essere generato da qualsiasi browser che segue le specifiche di rendering del W3C.
\end{document}