\documentclass[a4paper,12pt,oneside]{book}
\usepackage{graphicx}
\usepackage{fancyhdr}
\usepackage[font = scriptsize, bf]{caption}
\usepackage[italian]{babel}
\usepackage[latin1]{inputenc}
\usepackage[parfill]{parskip}
\usepackage{amsmath, amssymb}
\usepackage{moreverb}
\usepackage{algorithm}
\usepackage{algpseudocode}
\usepackage[usenames,dvipsnames]{color}
\usepackage{frontespizio}
\usepackage{url}
\usepackage{setspace}
\usepackage{eqparbox,array}
\usepackage{siunitx}
\usepackage{subfigure} 
\usepackage{wrapfig}
\usepackage{amsthm}

\renewcommand{\algorithmiccomment}[1]{  //\emph{\textcolor{Gray}{#1}}}


% Sistema i margini per lasciare pi� spazio nella zona di rilegatura
\addtolength{\oddsidemargin}{+1,0cm} 
\addtolength{\evensidemargin}{+1,0cm} 
\onehalfspacing

% Imposta lo stile della prima pagina del capitolo
\fancypagestyle{plain}
{
    \fancyhead{}
    \fancyfoot[LE,RO]{\thepage}
    \renewcommand{\headrulewidth}{0pt}
}

\DeclareMathOperator*{\argmax}{arg\,max}
\newcommand{\compInterfacciaDB}{Data Interface}
\newcommand{\compLoader}{Loader}
\newcommand{\compMatrix}{Matrix Creator}
\newcommand{\compTermsSel}{Terms Selector}
\newcommand{\compPosition}{Position Calculator}
\newcommand{\compClustering}{Clustering Component}
\newcommand{\compEvolution}{Evolution Discoverer}

\hyphenation{ti-me-win-dow}


\graphicspath{ {./Immagini/} }

\begin{document}

\begin{frontespizio}
		\Universita{Bari - ``Aldo Moro''}
		\Logo[3.5cm]{logo_uni}
		\Divisione{Dipartimento di Informatica}
		\Corso{\\Informatica e Tecnologie per la Produzione del Software}
		\Annoaccademico{2015-2016}
		\Titoletto{Tesi di laurea\\in\\Programmazione II}
		\Titolo{Applicazione di tecniche di Word Embedding e Text Mining per il Clustering di pagine in un grafo Web}
		\Candidato{Andrea Del Fante}
		\NCandidato{Laureando}
		\Relatore{Prof. Michelangelo Ceci}
		\Correlatore{Dott.ssa Pasqua Fabiana Lanotte}
		\Margini{3cm}{2cm}{2cm}{2cm}
	\end{frontespizio}
	
	\pagestyle{fancy}
	\fancyfoot{}
	\fancyfoot[LE,RO]{\thepage}
	\fancyhead{}
	\renewcommand{\headrulewidth}{0pt}
	\headheight = 15pt
	\frontmatter
	
	% indice
	\tableofcontents
	\listoftables
	\listoffigures
	\newpage

%******************************************************************
% Materiale iniziale
%******************************************************************

\cleardoublepage
\chapter*{Introduzione}

L'Informatica � diventata uno dei pilastri fondamentali su cui si regge la societ� moderna, ovvero quella dell'informazione. Ogni giorno vengono prodotti un gran numero di dati liberamente fruibili da chiunque, che vengono archiviati e organizzati in maniera automatica ed economica. Produrre nuova conoscenza significa eseguire un processo di elaborazione dei dati che possa arricchire il sapere pregresso. Aumentare il bagaglio della conoscenza significa, quindi, facilitare i processi decisionali.
\\\\
Questo problema � particolarmente sentito nel World Wide Web. Il Web � il pi� grande, eterogeneo e dinamico contenitore di sapere liberamente fruibile da chiunque. Queste caratteristiche, per�, rendono il processo di elaborazione dei dati e di produzione di nuova conoscenza una sfida impegnativa.
\\\\
Sorge, quindi, una nuova problematica: accedere all'enorme mole di dati archiviati nel Web in maniera veloce e mirata. Una possibile soluzione consiste nell'utilizzo di tecniche di Clustering su pagine Web, ossia assegnarle a gruppi in cui si trovano elementi appartenenti alla stessa tipologia semantica (per esempio pagine di processori, corsi e prodotti).
\\
Il Clustering di pagine Web non si propone come una nuova metodologia: in letteratura, infatti, esistono algoritmi che sfruttano o la struttura organizzativa di un sito Web o il testo contenuto nelle pagine, considerandole come documenti indipendenti tra loro.
\\\\
Ma nel contesto del Web, le pagine non possono essere trattate come documenti a se stanti, bisognerebbe piuttosto cercare di utilizzare l'informazione codificata nella struttura ad hyperlink del sito.
\\
La metodologia descritta in questa tesi non considera, infatti, un sito Web come una collezione di documenti testuali indipendenti tra loro, ma cerca di combinare informazioni relative al contenuto con quelle strutturali, in modo che due pagine Web vengano considerate simili se caratterizzate da una simile distribuzione di termini e abbiano una relazione di tipo diretta o indiretta, diretta se un hyperlink porta immediatamente ad una pagina, indiretta se vi sono pagine intermedie tra quella di partenza e di destinazione.
\\\\
Si definisce di seguito la struttura di questo lavoro di tesi.\\
Nel capitolo 1 ci si occuper� di descrivere i concetti essenziali per comprendere a pieno il campo applicativo in cui verr� effettuata la sperimentazione.
Nel capitolo 2 si parler� della frontiera attuale dell'Informatica in questo campo, descrivendo i risultati dei lavori correlati a quello descritto in questa tesi.
Nel capitolo 3 verranno descritti i passi eseguiti per questa sperimentazione, spiegandoli concettualmente e riportando gli algoritmi in pseudocodice.
Nel capitolo 4 si descriver� la sperimentazione effettuata, spiegando le metriche utilizzate, le configurazioni utilizzate e riportando le tabelle con i risultati utili per spiegare i valori ottenuti.
Nel capitolo 5, infine, si parler� dei possibili sviluppi futuri di questa attivit� di ricerca.

%******************************************************************
% Materiale principale
%******************************************************************
	% frontespizio
	
	\frontmatter	
	\mainmatter

	% Imposta lo stile di intestazione e pi� di pagina dei capitoli
	\fancyfoot{}
	\fancyhead{}
	\fancyhead[LE,RO]{\slshape \leftmark}
	\fancyfoot[LE,RO]{\thepage}
	\renewcommand{\headrulewidth}{1pt}
	\renewcommand{\chaptermark}[1]{
	\markboth{\thechapter.\ #1}{}}

\chapter{Informazioni latenti nel Web}
\label{cap:capitolo1}
\documentclass[a4paper,12pt,oneside]{book}
\usepackage[usenames]{color} %usato per il colore
\usepackage{amssymb} %maths
\usepackage{amsmath} %maths
%\usepackage[utf8x]{inputenc} %utile per scrivere direttamente in caratteri accentuati
\usepackage[latin1]{inputenc}
\usepackage[pdftex]{graphicx} 
\usepackage{filecontents}
\usepackage[italian]{babel}

\graphicspath{ {./../Immagini/} }


\begin{document}

Siamo sommersi dai dati.
\\\\
Ogni giorno viene generata una quantit� enorme di dati, anche da azioni della vita quotidiana: dalle applicazioni per smartphone alle carte di credito usate per gli acquisti, dai programmi usati sui computer ai sensori usati nelle infrastrutture intelligenti della citt�.\\
Nella stragrande maggioranza dei casi, questa enormit� di dati, chiamati Big Data, viaggia attraverso Internet, ed � possibile fruire di queste grandi quantit� di informazioni semplicemente esplorando il World Wide Web.\\
Tuttavia, data l'enormit� e l'eterogeneit� dei dati che si trovano nel Web, non � possibile utilizzarli direttamente: per farlo, si devono applicare delle metodologie per  analizzare ed estrarre l'informazione dal grafo del Web. Non solo, quindi, estrapolare l'informazione da una pagina, ma anche utilizzare la struttura ad hyperlink di cui il World Wide Web si compone.\\
In quest'ottica, il Data Mining si � evoluto.

\section{Data Mining}
Il Data Mining � l'insieme di tecniche che hanno come obiettivo l'estrazione del sapere o della conoscenza, partendo da grandi quantit� di dati. Queste tecniche e metodologie vengono usate sia in ambito industriale che scientifico.\\
Il termine significa letteralmente "estrazione di dati", la quale si divide in:
\begin{itemize}
	\item \textbf{estrazione}: l'informazione implicita, nascosta o formata da dati strutturati viene estratta per renderla immediatamente usabile;
	\item \textbf{esplorazione ed analisi}: vengono scoperti pattern significativi, per mezzo dei quali si estrae l'informazione significativa.
\end{itemize}
Con il termine pattern, nel contesto del Data Mining, si intende uno schema, una regolarit�, o, in generale, una rappresentazione sintetica dei dati \cite{cinecadatamining}.\\
%magari aggiungi qualche altra cosa

\paragraph{Natural Language Processing}
Il Natural Language Processing � il processo di trattamento automatico mediante un calcolatore delle informazioni scritte o parlate in una lingua naturale.\\
La difficolt� che caratterizza questo processo � l'elevato numero di ambiguit� che caratterizza il linguaggio umano: per questo motivo � stato diviso in quattro fasi, o sottoprocessi:
\begin{itemize}
	\item \textbf{analisi lessicale}: in questa fase avviene la scomposizione della sequenza di caratteri, chiamata espressione linguistica, in token (o parole);
	\item \textbf{analisi grammaticale}: in questa fase avviene l'associazione di ciascuna parola ad una parte del discorso;
	\item \textbf{analisi sintattica}: in questa fase avviene il parsing dei token e viene generato un albero di parser (parse tree);
	\item \textbf{analisi semantica}: in questa fase viene assegnato un significato al parse tree, la quale provvede alla disambiguazione l'espressione linguistica, ovvero ad assegnare un significato tra quelli disponibili.
\end{itemize}

\paragraph{Text Mining}
Il Text Mining, riferito anche come Text Data Mining o Text Analysis, � l'applicazione delle tecniche e metodologie del Data Mining ai testi. L'obiettivo � simile al Data Mining: estrarre informazioni latenti in documenti e testi analizzando ed esplorando dei pattern significativi.\\
Utilizzando il Natural Language Processing, � possibile estrarre informazioni incapsulate nei testi, le quali potrebbero essere potenzialmente utili.
%qui devi aggiungere qualche altra cosa, assolutamente 

\section{Web Mining}
Con l'espressione Web Mining ci si riferisce all'applicazione di procedure analoghe per estrarre automaticamente informazioni dalle risorse presenti nel Web, sia documenti che servizi \cite{webminingmalerba}. In altri termini, � l'applicazione delle procedure di Data Mining per scoprire pattern dal World Wide Web ed estrarre l'informazione. La conoscenza viene estratta dal contenuto, dalla struttura e dall'uso del Web.\\
In \cite{webminingmalerba} viene spiegato come l'obiettivo del Web Mining viene diviso in vari sotto-obiettivi:
\begin{itemize}
	\item \textbf{Scoperta di risorse}: gli strumenti per la scoperta di risorse, che vengono chiamati Spider, ovvero Web Robot, scandiscono milioni di documenti Web e costruiscono indici di ricerca in base alle parole che si trovano negli stessi;
	\item \textbf{Estrazione di informazioni}: i testi, che sono scritti in linguaggio naturale, vengono trasformati in rappresentazioni strutturate predefinite, dette template, che rappresentano un estratto dell'informazione presente nel testo;
	\item \textbf{Generalizzazione}: i processi di navigazione nel web devono essere generalizzati.
\end{itemize}
Il Web Mining pu� essere suddiviso in tre distinte categorie, in base al tipo di dato da estrarre: Web Usage Mining, Web Structure Mining e Web Content Mining.\\ Di seguito viene presentata la tassionomia delle varie tipologie di Web Mining.

\begin{figure}[h]
	\centering
	\includegraphics[width = 95mm]{webminingtax}
\end{figure}

\paragraph{Web Usage Mining}
Il Web Usage Mining � l'applicazione delle tecniche di Data Mining per la scoperta di pattern e informazioni utili attraverso l'analisi dei log, i quali sono immagazzinati nei Web server o da sistemi che tracciano le attivit� degli utenti.\\
L'obiettivo di questo campo � la profilazione dell'utente, ovvero analizzare i suoi comportamenti sul web sia per comprendere quali sono i suoi reali bisogni, sia per offrire dei servizi che possano soddisfare tali necessit� e personalizzare l'esperienza Web.\\
Questo tipo di Data Mining viene usato in disparati campi, che vanno dalle aziende alle agenzie governative: i siti di e-commerce usano questo tipo di tecnologia per presentare all'utente prodotti per i quali potrebbe essere interessato; le agenzia governative, invece, hanno usato il Web Mining per classificare minacce e attentati terroristici.\\
Alcuni, per�, criticano questa tecnologia: il problema etico di cui pi� si parla � la violazione della privacy.

\paragraph{Web Structure Mining}
Il Web Structure Mining � un processo di analisi della struttura di un sito web, il quale viene considerato come un grafo i cui nodi sono le pagine e gli archi sono gli hyperlinks tra le pagine. Si divide in:
\begin{itemize}
	\item Estrazione di schemi dagli hyperlinks, in cui un hyperlink � un arco tra due pagine web;
	\item Estrazione della struttura del documento, ovvero l'analisi della struttura ad albero basata su HTML ed XML.
\end{itemize}
Quindi, questo tipo di Web Mining pu� essere effettuato sia a livello di documento web (intra-pagina), sia a quello di hyperlinks (inter-pagina).\\
Basata sulla topografia degli hyperlinks, Web Structure Mining pu� categorizzare le pagine web e generare informazioni come la similarit� e le relazioni tra i differenti siti Web \cite{Nilima}.\\
Tra i pi� importanti algoritmi che appartengono a questa tipologia troviamo Page Rank \cite{pagerankstanford} e HITS \cite{Kleinberg99}, i quali sfruttano la struttura ad hyperlink del Web per assegnare un rank alle pagine, ovvero per restituirle in ordine di importanza relativamente ad una determinata query.

\paragraph{Web Content Mining}
L'ultimo tipo di Web Mining � il Web Content Mining, il quale viene usato per cercare informazioni utili dal contenuto delle pagine Web. Con il termine contenuto ci si riferisce a collezioni di testi, immagini, audio, video, o record strutturati che sono incapsulati in liste e tabelle.
Nel campo della ricerca, � stato applicato il Text Mining, che ha permesso di migliorare le attivit� di mining sui testi grazie al Natural Language Processing.\\
Per estrarre il sapere da contenuti pi� complessi, come le immagini, le tecniche di Web Content Mining sono molto limitate\cite{webmining21cap}.

\section{Rappresentazioni vettoriali di pagine Web}
Il campo del Web Mining, come visto in precedenza, � una intersezione di molte discipline, quali Information Retrieval, Web Mining, Data Mining e Machine Learning. 

\subsection{Word2Vec}

\subsection{Line}

\subsection{Doc2Vec}

\subsection{Word space model}

\section{Clustering}
Con il termine clustering si intende l'insieme di tecniche che hanno come scopo quello di selezionare e raggruppare, da una collezione di dati, elementi omogenei, avendo come base la somiglianza tra gli stessi. La somiglianza tra gli elementi � concepita in termini di distanza di uno spazio multidimensionale. La bont� della similarit�dipende fortemente dalla funzione che si usa per calcolare la distanza tra gli elementi.

\subsection{Approcci di clustering}
L'operazione di clustering � essenzialmente la creazione di un insieme di clusters, cio� un insieme di insiemi, che generalmente contengono tutti gli elementi iniziali. Si possono usare varie classi di approcci per effettuare clustering su un determinato insieme di dati iniziali. Alcune di queste sono:
\begin{itemize}
    \item Hard clustering o soft clustering
    \item Partizionali o gerarchici
\end{itemize}

\paragraph{Hard clustering e soft clustering}
Questi algoritmi attuano un approccio secondo cui un elemento pu� essere assegnato ad un solo cluster o a pi� cluster. Con hard clustering intendiamo che l'algoritmo assegna un elemento ad uno ed un solo cluster; con soft clustering, invece, l'elemento pu� essere assegnato a pi� cluster con gradi di appartenenza diversi.

\paragraph{Clustering partizionale}
Gli algoritmi di clustering partizionali creano una divisione delle osservazioni minimizzando una certa funzione di costo:
\begin{equation}
   {\textstyle \sum_{j=1}^{k}} E(C_j)
\end{equation}
dove $k$ � il numero desiderato di cluster, $C_j$ � il j-esimo cluster ed $E : C \to \mathbb{R^+}$ � la funzione di costo associata al singolo cluster. L'algoritmo pi� famoso che fa parte di questa categoria � K-Means.

\paragraph{Clustering gerarchico}
Gli algoritmi facente parti di questa categoria non suddividono lo spazio, bens� costruiscono una gerarchia di cluster. In questa strategia rientrano due sottotipi:
\begin{itemize}
\item \textbf{Aggregativo}: tale approccio considera n cluster per n elementi, cio� ogni elemento viene considerato un cluster a s�. Successivamente, l'algoritmo unisce tutti i cluster pi� vicini. Viene anche chiamato bottom-up.
\item \textbf{Divisivo}: tale approccio ragiona in maniera opposta rispetto al precedente, poich� tutti gli elementi vengono considerati come un unico cluster, e l'algoritmo deve dividere il cluster in insiemi aventi dimensioni inferiori. Questa metodologia viene anche chiamata top-down.
\end{itemize}
Durante l'aggregazione degli elementi � necessario usare una funzione che permette di calcolare la similarit� (o meglio dire la distanza) tra due cluster: questo permette all'algoritmo di unire i cluster simili. 

\subsection{Funzioni (o misure) di distanza}
A seconda dell'approccio utilizzato, vi sono delle funzioni (o misure) che permettono di calcolare la distanza tra due cluster. Viene molto usato dagli algoritmi di clustering gerarchico per calcolare la similari� tra i cluster e per unire, eventualmente, i cluster simili. Le funzioni di distanza usate da questo tipo di clustering sono: \textbf{single-link proximity}, \textbf{average-link proximity}, \textbf{complete-link proximity} e la \textbf{distanza tra centroidi}.

\paragraph{Single-link proximity}
Questa funzione calcola la distanza tra due cluster come la distanza minima tra elementi appartenenti a cluster differenti.
\begin{equation}
    D(C_i, C_j) = min_{x \in C_i, y \in C_j} d(x, y)
\end{equation}
\begin{figure}[h]
	\centering
	\includegraphics[width = 60mm]{clustering_single}
\end{figure}


\paragraph{Average-link proximity}
Questa funzione calcola la distanza tra due cluster come la media delle distanze tra i singoli elementi.
\begin{equation}
    D(C_i, C_j) = 1 / (|C_i||C_j|) \sum_{x \in C_i, y \in C_j} d(x, y)
\end{equation}
\begin{figure}[h]
	\centering
	\includegraphics[width = 60mm]{clustering_average}
\end{figure}


\paragraph{Complete-link proximity}
Questa funzione calcola la distanza tra i due cluster considerando la distanza massima tra gli elementi appartenenti ai due cluster.
\begin{equation}
    D(C_i, C_j) = max_{x \in C_i, y \in C_j} d(x, y)
\end{equation}
\begin{figure}[h]
	\centering
	\includegraphics[width = 60mm]{clustering_complete}
\end{figure}

\paragraph{Distanza tra centroidi}
Questa, invece, � la distanza tra i due cluster prendendo in considerazione i centroidi degli stessi.
\begin{equation}
    D(C_i, C_j) = d\left ( \hat{c_i}, \hat{c_j} \right )
\end{equation}
\begin{figure}[h]
	\centering
	\includegraphics[width = 60mm]{clustering_centroid}
\end{figure}


Nei casi precedenti, $d(x, y)$ indica una qualsiasi funzione distanza su uno spazio metrico, le quali possono essere:
\begin{itemize}
\item \textbf{Distanza euclidea}: chiamata anche norma 2, � la distanza calcolata tra due punti, la quale pu� essere misurata su uno spazio multidimensionale. Siano $P = (p_1, p_2, ..., p_n)$ e $Q = (q_1, q_2, ..., q_n)$ due punti, la distanza sar�:
    \begin{equation}
        \sqrt{(p_1 - q_1)^2 + (p_2 - q_2)^2 + ... + (p_n - q_n)^2} = \sqrt{\sum_{k=1}^{k} (p_k - q_k)^2}
    \end{equation}

\item \textbf{Distanza di Manhattan}: chiamata anche geometria del taxi o norma 1, � la distanza tra due punti calcolata come la somma del valore assoluto delle differenze delle loro coordinate. Siano $P_1 = (x_1, y_1)$, $P_2 = (x_2, y_2)$ due punti, la distanza sar�:
    \begin{equation}
        L_1(P_1, P_2) = |x_1 - x_2| + |y_1 - y_2|
    \end{equation}

\item \textbf{Norma uniforme}
\item \textbf{Distanza di Mahalanobis}
\item \textbf{Coseno di similarit�}: tecnica euristica usata per misurare la distanza tra due vettori, che viene effettuata calcolando il coseno dell'angolo compresovi, che hanno l'origine coincidente con quello del sistema di assi e passano per i rispettivi elementi. Il valore risultante pi� sar� vicino ad 1, pi� i due elementi sono simili tra loro. Siano A e B due vettori di attributi numerici,
    \begin{equation}
        \cos(\theta) = \frac{AB}{||A|| ||B||}
    \end{equation}

\item \textbf{Distanza di Hamming}: misura il numero di sostituzioni necessarie per convertire una stringa nell'altra, oppure pu� essere vista come un reporting del numero degli errori che hanno trasformato una stringa nell'altra. La distanza di Hamming tra 10{\color{red}1}1{\color{red}1}01 e 10{\color{red}0}1{\color{red}0}01 � 2; oppure tra 2{\color{red}14}3{\color{red}8}96 e 2{\color{red}23}3{\color{red}7}96 � 3.
\end{itemize}

\subsection{Algoritmi usati}
In questa sezione vengono descritti gli algoritmi di clustering che sono stati usati sui dataset della sperimentazione. Questi sono stati usati per verificare la bont� dell'operazione di clustering, partendo da elementi che sono stati appresi mediante algoritmi di apprendimento differenti, la quale � stata analizzata mediante apposite metriche.

\paragraph{K-Means}
K-Means � un algoritmo di clustering di tipo partizionale, in cui ogni cluster viene identificato mediante un centroide. Si basa sull'algoritmo di Lloyd e consiste in 3 step. Il primo step consiste nella scelta dei centroidi iniziali, i quali saranno K elementi, casuali o usando informazioni euristiche, scelti dal dataset. Successivamente, l'algoritmo assegna per ogni elemento il centrine pi� vicino e crea nuovi centroidi dalla media di tutti i campioni, assegnati ai centroidi precedenti. Si ripete questa fase finch� l'algoritmo non converge.
Il pregio principale di questo algoritmo � che converge molto velocemente: si � analizzato, infatti, che il numero di iterazioni che l'algoritmo esegue � minore del numero di elementi del dataset.
K-means, per�, pu� essere molto lento nel caso peggiore e non garantisce il raggiungimento dell'ottimo globale: la bont� della soluzione dipende dal set di cluster iniziale. Inoltre, un altro svantaggio � che l'algoritmo richiede, in input, il numero dei cluster.

\paragraph{HDBScan}
HDBScan � un algoritmo di clustering che estende DBScan, rendendolo di tipo gerarchico. Si parte in maniera simile a DBScan: lo spazio viene trasformato a seconda della densit� e viene effettuato su di esso una prossimit� a single-link.
Invece di richiedere come input il parametro $\epsilon$, che viene usato da DBScan per considerare gli elementi del vicinato appartenenti al cluster, viene creato un albero, il quale viene usato per selezionare i cluster pi� stabili e persistenti.
Al posto di $\epsilon$, quindi, viene richiesta la dimensione minima dei cluster per determinare quali gruppi non devono essere considerati come cluster, oppure per dividerli e formare nuovi cluster.
Questo algoritmo � molto efficace ed � il pi� veloce, sia di DBScan che di K-Means.

\bibliographystyle{plain}
\bibliography{./../Bibliografia}                % database di biblatex 

\end{document}

\chapter{Stato dell'Arte}
\label{cap:capitolo2}
\documentclass[a4paper,12pt,oneside]{book}
\usepackage[usenames]{color} %usato per il colore
\usepackage{amssymb} %maths
\usepackage{amsmath} %maths
\usepackage[latin1]{inputenc}
\usepackage[pdftex]{graphicx} 
\usepackage{filecontents}
\usepackage[italian]{babel}
\usepackage{setspace}

\graphicspath{ {./../Immagini/} }

\begin{document}

\addtolength{\oddsidemargin}{+1,0cm} 
\addtolength{\evensidemargin}{+1,0cm} 
\onehalfspacing

%\tableofcontents
%\listoftables
%\listoffigures
%\newpage

% related work, dove � arrivata la scienza e i lavori che giustificano quello che ho fatto, il perch�
L'applicazione delle tecniche di Clustering su pagine di siti Web non � un nuovo campo di ricerca. La maggior parte delle metodologie che si trovano in letteratura sono state usate per raggruppare le pagine Web.
\\\\
Tuttavia queste ricerche sono state indirizzate sul processo di Clustering di pagine provenienti da diversi siti Web, trascurando le pagine di uno specifico sito. Gli hyperlink, infatti, hanno significati diversi in base al dominio di destinazione: se la pagina puntata si trova nello stesso sito Web, il collegamento avr� funzione di organizzazione dei contenuti del sito; altrimenti se il collegamento punta ad una pagina esterna sar� indirizzato a pagine che probabilmente avranno contenuti simili.
\\\\
Gli algoritmi di Clustering esistenti si classificano in quattro categorie in base alle informazioni che questi usano per raggruppare le pagine Web:
\begin{itemize}
	\item \textbf{Algoritmi di Clustering basati sul contenuto testuale}. Questa tipologia di algoritmi considerano le pagine Web come dei documenti testuali indipendenti. Questo � il caso di \cite{Zamir,Chehreghani,Haveliwala,Anami}, dove la distribuzione delle parole � usata per scoprire insiemi appropriati di pagine Web correlate. Il vantaggio di questo approccio � che molti strumenti di clustering, basati sul modello dello spazio vettoriale, possono essere direttamente applicabili. Lo svantaggio � che questa tipologia fallisce quando devono essere appresi modelli accurati, a causa della natura non controllata ed eterogenea dei contenuti delle pagine Web. Infatti, i tradizionali algoritmi di Clustering si basano sull'assunto che i documenti testuali condividono stili di scrittura consistendi, dando abbastanza informazioni contestuali, sono chiari e completamente non strutturati, e sono indipendenti e identicamente distribuiti. Queste limitazioni sono pi� ovvie per il Clustering di pagine Web di differenti siti o il cui contesto � creato in maniera collaborativa. In questo fatto, infatti, le pagine Web aventi stesso argomento potrebbero essere contestualmente differenti: in altre parole, potrebbero avere un contenuto informativo simile immerso, per�, in elementi Web di diverse regole semantiche (i.e. tabelle o menu di navigazione) e di differenti funzionalit� (i.e. link, pulsanti, immagini).
	\item \textbf{Algoritmi di Clustering basati sui Web log}.
	\item \textbf{Algoritmi di Clustering basati sulla struttura HTML}. Questi tipi di algoritmi di Clustering hanno il vantaggio di considerare sia l'informazione strutturale che visuale inserita nei tag HTML, che viene ignorata dall'approccio testuale. 
	\item \textbf{Algoritmi di Clustering basati sulla struttura ad hyperlink}.
\end{itemize}
Le pagine Web, a differenza dei documenti, sono caratterizzate da propriet� strutturali come tag HTML, propriet� visuali ed hyperlink, che definiscono la loro rappresentazione strutturale. E' stato provato da \cite{crescenzi,bohunsky,lin,zhu}, in cui l'informazione strutturale fornisce una differente e e complementare informazione rispetto alla rappresentazione testuale.

%\bibliographystyle{plain}
%\bibliography{./../Bibliografia}                % database di biblatex 

\end{document}

\chapter{Metodologia}
\label{cap:capitolo3}
Il Clustering delle pagine di un sito Web � un processo fondamentale nel Web Mining utile a valutare l'interazione tra le pagine, organizzare i contenuti del sito e capire come questo sia stato strutturato. 
\\
Una pagina Web � composta da varie rappresentazioni (Sezione \ref{webpagerapresentation}) che la caratterizzano e la diversificano dalle altre. Attualmente, queste propriet� vengono sfruttate dagli algoritmi di Clustering in maniera indipendente.
\\
Gli obiettivi di questa tesi, descritti nella Sezione \ref{aimthesis}, sono stati raggiunti mediante l'utilizzo di una metodologia composta da 3 passi:
\begin{itemize}
	\item \textit{Crawling del sito Web}
	\item \textit{Costruzione del Dataset}
	\item \textit{Clustering delle pagine Web}
\end{itemize}
Di seguito, si descrivono in dettaglio le fasi della metodologia.

\section{Web Crawling}
Le propriet� che caratterizzano le pagine Web rendono complicato il processo di estrazione di informazioni, soprattutto nel caso in cui i contenuti vengono generati dinamicamente. Per analizzare i contenuti della rete e delle pagine Web si utilizza un software automatizzato chiamato Web Crawler. Tipicamente, questo programma viene usato per molti altri scopi, come l'indicizzazione di pagine Web. I motori di ricerca sfruttano tali indici per aumentare l'efficienza delle interrogazioni che gli utenti effettuano durante la navigazione.
\\
La ragione principale dell'uso dei Crawler � che il World Wide Web non � un contenitore centralizzato: pu� essere visto, infatti, come un insieme di siti Web che forniscono differenti servizi.
\\\\
L'algoritmo di Crawling � relativamente semplice: dato un insieme di URL di pagine Web, vengono scaricate tutte le pagine associate all'indirizzo URL, estratti gli hyperlinks e, iterativamente, scaricate le pagine associate a questi link. Il loro contenuto verr� analizzato e salvato per essere successivamente indicizzato. Nonostante l'apparente semplicit� di questo algoritmo, l'attivit� di Web Crawling � caratterizzata da obiettivi inerenti \cite{crawling}:
\begin{itemize}
	\item \textbf{Scalabilit�}. Il Web � molto grande e in continua evoluzione. I Crawler che cercano una copertura ampia ed aggiornata devono raggiungere throughput (prestazioni) estremamente alti. Per riuscirci, bisogna risolvere problematiche ingegneristiche complesse. I moderni motori di ricerca impiegano migliaia di computer e decine di collegamenti di rete ad alta velocit�.
	\item \textbf{Compromessi per la selezione dei contenuti}. Persino i Crawler con un alto throughput non sono in grado di scandire l'intero Web o tenere traccia di tutti i cambiamenti. Per questo, il processo di Crawling viene effettuato in maniera selettiva con un attento e controllato ordine. Gli obiettivi sono di acquisire velocemente i contenuti aventi un alto valore, di assicurare la copertura di tutti i contenuti scelti, di ignorare quelli a bassa qualit�, irrilevanti, ridondanti e dannosi e di mantenerli aggiornati. Il Crawler deve rispettare alcuni vincoli come il numero massimo di visite per sito e le pagine da scartare durante l'analisi.
	\item \textbf{Obblighi sociali}. I Crawler dovrebbero essere dei "buoni cittadini" del Web: non devono sovraccaricare i siti che scandiscono. Infatti, senza i giusti meccanismi, un throughput alto pu� inavvertitamente provocare un attacco Denial of Service (DOS), sovraccaricando il server Web e rendendolo inaccessibile.
	\item \textbf{Avversari}. Alcuni siti Web cercano di iniettare contenuti inutili o fuorvianti nel corpus assemblato dai Crawler. Questo comportamento � spesso motivato da incentivi finanziari, per esempio per indirizzare male il traffico verso siti commerciali.
\end{itemize}

\begin{figure}[h]
	\centering
	\includegraphics[width = 100mm]{WebCrawlerArchitecture}
	\caption{Architettura di un web Crawler}
	\label{crawlerarchitecture}
\end{figure}

\subsection{Crawling delle pagine di un sito Web}
Ai fini sperimentali � stato utilizzato un Web Crawler che estrae gli hyperlink di un sito Web per effettuare successivamente l'operazione di Clustering delle pagine associate a tale sito. Un sito Web pu� essere formalmente descritto come un grafo orientato $G = (V, E)$, dove $V$ � l'insieme delle pagine appartenenti al sito ed $E$ � l'insieme degli hyperlink. In molti casi, la homepage $h$ di un sito rappresenta il punto di inizio, tecnicamente espresso come un grafo orientato radicato (albero), che consente l'esplorazione da parte degli utenti.
\\
Il grafo del sito Web pu� essere ricco di link rumore (e.g. hyperlink scorciatoia) che non sono rilevanti nel processo di Clustering \cite{crescenzi} e vengono esclusi dal Crawling. In pi�, la struttura del sito Web � codificata in sistemi di navigazione che offrono una visuale della sua organizzazione. Questi sistemi vengono implementati come collezioni di link che hanno lo stesso dominio e condividono il layout e le propriet� visuali.

\begin{algorithm}[h]
	\caption{crawlingWebsite(homepage)}
	
	\begin{algorithmic}
		\State \textbf{Input:} URL homepage
		\State \textbf{Output:} Set$<$(URL, URL)$>$ E; Set$<$(URL, String)$>$ V
		\State $frontier \gets Set()$
		\State $Q \gets Queue(homepage)$
		
		\Repeat
			\State $currentPage \gets Q.dequeue()$
			\State $text \gets currentPage.getText()$
			\State $V.add((currentPage, text))$
			\State $webLists \gets extractWebLists(currentPage)$
			
			\For{\textbf{each} $a \in b$}
				\State $pagesToAnalyze \gets list.filterNot(page \to frontier.contains(page))$
				\State $Q.enqueue(pagesToAnalyze$
				\State $frontier.add(pagesToAnalize)$
				
				\For{\textbf{each} $u \in pagesToAnalyze$}
					\State $E.add((currentPage, u))$
				\EndFor
			\EndFor
		\Until{$!Q.empty()$}
		\State \textbf{return} $(V, E)$
	\end{algorithmic}
\end{algorithm}

La soluzione utilizzata per indicizzare un sito Web � stata quella di sfruttare il concetto di \textbf{lista Web}.
\\
Per definizione, una lista Web � una collezione di due o pi� elementi web che hanno struttura HTML simile, visualmente adiacenti ed allineati sulla pagina renderizzata. Questo allineamento pu� essere visto sia lungo l'asse x (i.e. una lista verticale), sia lungo l'asse y (i.e. una lista orizzontale), o ancora in maniera mista (i.e. griglia).

\begin{figure}[h]
	\centering
	\includegraphics[width = 75mm]{weblistexample}
	\caption{Esempio di liste Web}
	\label{weblistfigure}
\end{figure}

In Figura \ref{weblistfigure} vengono mostrate le liste Web, usate per guidare il processo di Crawling. I link inseriti nel box $A$ sono stati esclusi perch� il loro dominio � differente da quello della homepage. Il risultato del Crawler � un sottografo $G' = (V', E')$, dove $V' \subseteq V$ ed $E' \subseteq E$.

\subsection{Normalizzazione degli URL}
Una volta effettuato il processo di Crawling ed estratti gli URL, si procede con la normalizzazione degli stessi. Questo � un processo in cui gli URL vengono modificati e standardizzati in maniera consistente. Il suo obiettivo � poter determinare se due URL, che sono sintatticamente differenti, possono essere equivalenti.
\\
I Crawler effettuano un qualche tipo di normalizzazione degli URL in modo da evitare che il processo di Crawling non vada ad analizzarli pi� volte.
\\\\
\texttt{http://www.facebook.com/}
\\
\texttt{facebook.com/}
\\\\
\texttt{http://208.77.188.166/}
\\
\texttt{http://www.example.com/}
\\\\
La normalizzazione di URL, come si evince dagli esempi sopra riportati, pu� comprendere sia la rimozione del protocollo ("http://") e della stringa "www", oppure la sostituzione dell'indirizzo IP con il nome del dominio. E' bene sottolineare come le due coppie in analisi puntino a due siti Web, rispettivamente \textit{Facebook} ed \textit{example}.
\\
Ci sono diverse modalit� di normalizzazione che possono essere effettuate, fra cui la conversione degli URL in minuscolo e la rimozione del "." e ".." per portare gli URL da assoluti a relativi, aggiungere slash finali al componente di percorso non vuoto.
\\
Per la sperimentazione si � scelto di normalizzare gli URL eliminando la dicitura del protocollo ("http://" o "https://"), del "www" e dello slash finale.

\section{Costruzione del dataset}
Il Crawler, a seguito dell'analisi del sito, produce un grafo delle pagine Web ed il contenuto testuale di ogni pagina esplorata. Il grafo del sito Web servir� per generare le sequenze attraverso i Random Walks.
\\
Di seguito verr� spiegato il concetto di Random Walk e come questi sono stati utilizzati ai fini della sperimentazione.

\subsection{Random Walks}
\label{randomwalkssection}
Un Random Walk, o passeggiata aleatoria, � la formalizzazione dell'idea di effettuare passi successivi in direzioni casuali. Dal punto di vista matematico � il processo stocastico pi� semplice, ovvero quello markoviano, nel quale la probabilit� che determina il passaggio ad uno stato, dipende solo dallo stato immediatamente precedente, e non dal come si � giunti a tale stato.
\\\\
In una passeggiata aleatoria monodimensionale si studia il moto di una particella puntiforme vincolata a muoversi lungo una retta nelle due direzioni consentite. Ad ogni movimento, questa si sposta a caso o a destra, con una probabilit� fissata \textbf{p}, oppure a sinistra, con una probabilit� \textbf{1 - p}, ed ogni passo � di lunghezza uguale e indipendente dagli altri.

\begin{figure}[ht!]
	\centering
	\includegraphics[width = 70mm]{randomwalkmono}
	\caption{Esempio di otto Random Walks in una dimensione}
	\label{randomwalkmono}
\end{figure}

In una passeggiata aleatoria bidimensionale si studia il moto di una particella vincolata a muoversi sul piano spostandosi casualmente ad ogni passo a destra, a sinistra, in alto o in basso con probabilit� \textbf{1/2}. In particolare, ad ogni passo, la particella pu� compiere un movimento lungo una delle quattro diagonali con probabilit� \textbf{1/4}. Ma qual � la probabilit� che la particella torni al punto di partenza? In questo caso, la particella, che � libera di camminare casualmente con uguale probabilit� nelle quattro direzioni, torner� infinite volte al punto di partenza.

\begin{figure}[ht!]
	\centering
	\includegraphics[width = 70mm]{randomwalkbi}
	\caption{Esempio di Random Walks in due dimensioni}
	\label{randomwalkbi}
\end{figure}

In una passeggiata aleatoria tridimensionale si studia il moto di una particella vincolata a muoversi nello spazio spostandosi casualmente ad ogni passo a destra, a sinistra, in alto, in basso, in su o in gi� con probabilit� \textbf{1/2}. In pratica, ad ogni passo pu� compiere un movimento lungo una delle otto diagonali con probabilit� \textbf{1/8}. Come nel caso precedente, � stata calcolata la probabilit� che la particella torni prima o poi al punto di partenza ed � pari a \textbf{0,239}.

\begin{figure}[ht!]
	\centering
	\includegraphics[width = 70mm]{randomwalktri}
	\caption{Esempio di Random Walks in tre dimensioni}
	\label{randomwalktri}
\end{figure}

Il concetto di Random Walk � stato applicato nei campi pi� disparati, alcuni dei quali sono:
\begin{itemize}
	\item \textbf{Economia finanziaria}. I Random Walk sono stati usati per modellare i prezzi delle azioni sui mercati azionari, tassi di cambio di moneta e materie prime.
	\item \textbf{Genetica}. Descrivono le propriet� statistiche della deriva generica, ovvero una componente dell'evoluzione di una specie dovuta a fattori casuali, che pu� essere studiata con metodi statistici.
	\item \textbf{Fisica}. Usati come modelli semplificati per studiare il moto browniano, ovvero il moto della particelle presenti in fluidi che � osservabile al microscopio.
	\item \textbf{Ecologia matematica}. Usati per descrivere i movimenti dei singoli animali, i processi di diffusione della materia, o ancora per la dinamica della popolazione. Quest'ultimo studia i cambiamenti del numero di individui, della densit� e della struttura di una o diverse popolazioni. Inoltre analizza i processi biologici e ambientali che influenzano queste trasformazioni.
	\item \textbf{Informatica}. Usati per stimare la dimensione del Web, ovvero il numero di pagine e di siti che fanno parte del World Wide Web.
\end{itemize}

\subsection{Generazione delle sequenze}
Una passeggiata aleatoria sulla struttura ad hyperlink di un sito Web si basa sull'idea che le connessioni tra nodi (i.e. le pagine del sito) presentano delle informazioni latenti circa la loro correlazione. Per catturarle si � deciso di utilizzare un Random Walker, una componente incaricata di effettuare le passeggiate aleatorie sul grafo di un sito Web. Questa scelta � motivata dal fatto che le metodologie selettive per esplorare un sito Web, che derivano dalla teoria dei grafi, prevedono l'esplorazione di tutte le possibili opzioni per arrivare alla soluzione. Ma tali tecniche sono difficilmente computabili in quanto ricadono nella classe di complessit� NP-completa.
\\
Il Random Walker, anche se permette di avere buone approssimazioni nell'esplorazione del grafo del sito, presenta una problematica: potrebbe capitare che la pagina che sta analizzando non presenta link al suo interno. La soluzione pi� diffusa � quella di effettuare un "salto" verso una qualsiasi altra pagina. Nel nostro caso, invece, dato che le sequenze da generare devono avere una lunghezza massima fissata prima della generazione, se l'attraversatore casuale incontra una pagina priva di hyperlink, allora si blocca semplicemente. La sequenza finale quindi, risulter� pi� piccola. Questa scelta � stata presa poich� l'informazione cercata nasce da percorsi reali di navigazione. Inoltre non vi � la necessit� di una lunghezza obbligatoria da rispettare, in quanto le sequenze possono essere viste come frasi di un testo, dove le parole sono gli URL.

\begin{algorithm}[h]
	\caption{rwrGeneration(rwrLength, dbLength, G, $\alpha$)}
	\label{rwrGeneration}
	
	\begin{algorithmic}
		\State \textbf{Input:} int rwrLength \Comment{numero di passi massimo}
		\State \textbf{Input:} int dbLength \Comment{numero di frasi}
		\State \textbf{Input:} Graph G \Comment{il grafo del sito Web}
		\State \textbf{Input:} float $\alpha$
		\State \textbf{Output:} List$<$List$<$URL$>$$>$ randomWalks
		
		\For{\textbf{each} $i \in Range(0, dbLength)$}
			\State $w \gets List()$
			\State $w[0] \gets G.getRandomVertex()$

			\For{\textbf{each} $j \in Range(1, rwrLength)$}
				\State $\lambda \gets Math.random()$
				\If{$\lambda > \alpha$}
					\State $w[j] \gets G.getRandomOutlink(w[j-1])$
				\Else
					\State $w[j] \gets w[0]$
				\EndIf
			\EndFor
			
			\State $randomWalks.add(w)$
		\EndFor
		
		\State \textbf{return} $randomWalks$
	\end{algorithmic}
\end{algorithm}

Per motivi di sperimentazione sono stati implementati tre tipi diversi di Random Walk, utilizzabili modificando i parametri di esecuzione dell'Algoritmo \ref{rwrGeneration}.

\begin{itemize}
	\item \textbf{Random Walk standard}. Il caso standard prevede che si parta da un nodo casuale del grafo e si segua ogni volta un arco a caso fra quelli disponibili, fino al raggiungimento della lunghezza prefissata o all'impossibilit� di proseguire.
	
	\item \textbf{Random Walk con partenza da homepage}. Con quest'altra modalit� si ha una partenza fissata. Si parte, quindi, da un nodo prefissato del grafo, generalmente la homepage del sito Web in analisi, in modo da esplorare pi� percorsi possibili.
	
	\item \textbf{Random Walk attraverso le Liste}. Questo � il caso in cui si pu� eseguire uno delle due tipologie di Random Walks viste in precedenza, ma aventi il vincolo delle liste: l'algoritmo, quindi, operer� solo su un sottoinsieme di quello prodotto dalla metodologia scelta.
\end{itemize}

Quindi, i Random Walks prodotti sono stati trattati come frasi, in cui le parole sono i codici univoci degli URL. L'utilizzo di tali codici ha permesso di ridurre lo spazio in memoria e i tempi di elaborazione.
\\
Le frasi generate dal Random Walk standard sono state sfruttate da Word2Vec per apprendere la struttura del grafo del sito Web. Inoltre, per raggiungere uno degli obiettivi di questa tesi, � stata modificata l'implementazione di Word2Vec e successivamente applicata sui Random Walk con partenza fissa (dalla homepage). La stessa cosa � stata effettuata utilizzando le Liste Web.
\\
Nella sezione successiva viene spiegata, in maniera approfondita, la modifica di Word2Vec.

\subsection{Modifica dell'implementazione di Word2Vec}
\label{modifiedSkipgram}
Un aspetto particolare preso in considerazione in questa tesi � stato quello di chiedersi se, modificando in maniera opportuna l'algoritmo di Word Embedding Word2Vec, si potesse avere un miglioramento nel processo di Clustering delle pagine del grafo del sito Web in analisi.
\\
Per rispondere a questa domanda, si � deciso di analizzare e modificare il modello di apprendimento di Word2Vec Skip-Gram della libreria \texttt{deeplearning4j}. Questo modello consiste nel predire il contesto a partire da una parola. Per una argomentazione pi� approfondita si veda la Sezione \ref{word2vec}.
\\
La modifica di Skip-Gram consiste nel limitare l'apprendimento dell'algoritmo in maniera tale che venga analizzato solo il contesto sinistro, data una parola. E' importante sottolineare, inoltre, come questo modello sia stato ottimizzato con un valore chiamato \textbf{b}, che permette di aumentare o diminuire la finestra di contesto, dando pi� importanza alle parole pi� vicine a quella in analisi.
\\
Per la sperimentazione sono stati prodotti, quindi, due tipologie di modelli di Skip-Gram che analizzano solo il contesto sinistro: uno che utilizza il valore b ed un'altro che non lo usa.

\subsection{Scaling degli embeddings}
\label{scaling}
Una volta prodotti gli spazi vettoriali delle parole, sono state analizzate metodologie di Feature Scaling per capire la migliore da usare per la sperimentazione. 
\\
Il Feature Scaling � un metodo usato per standardizzare un intervallo di variabili indipendenti o features di dati. Viene anche chiamato normalizzazione di dati ed � generalmente effettuato nei passi di preprocessing di dati. Non solo, normalizzare i dati significa anche ridurre l'effetto degli Outlier. Il termine Outlier � usato in statistica per definire, in un insieme di osservazioni, un valore anomalo e aberrante (i.e. un valore chiaramente distante dalle altre osservazioni disponibili). Un numero consistente di Outlier nel campione in analisi pu� portare a risultati fuorvianti. Per esempio, se misurassimo la temperatura di 10 oggetti presenti in una stanza, la maggior parte dei quali risultasse avere una temperatura compresa fra i 20 e 25 gradi Celsius, allora il forno acceso, avente una temperatura di 350 gradi, sarebbe un dato aberrante.
\\\\
Le tecniche di Scaling pi� utilizzate sono:
\begin{itemize}
	\item \textbf{Z-score}. Con questa metodologia, i vettori vengono standardizzati ed avranno le propriet� di una distribuzione normale standardizzata, particolarmente utile nelle operazioni di stima statistica. Sono curve simmetriche con valori pi� concentrati verso il centro e meno nelle estremit� laterali. Un esempio � la curva di Gauss.
	\\
	Si avranno vettori aventi $\mu = 0$ e $\sigma = 1$, dove $\mu$ � la media e $\sigma$ � la deviazione standard dalla media. Gli Z-score sono calcolati come segue:
	\begin{equation}
		z = \frac{x - \mu}{\sigma}
	\end{equation}
	\item \textbf{Min-Max}. E' un approccio alternativo allo Z-score visto in precedenza. I dati vengono normalizzati usando un intervallo fissato, generalmente tra 0 ed 1, dove 0 � il valore minimo ed 1 quello massimo. Viene applicata la formula:
	\begin{equation}
		X_{norm} = \frac{X - X_{min}}{X_{max} - X_{min}}
	\end{equation}
	\item \textbf{L2}. Viene chiamato anche Normalizzazione Euclidea. Questo metodo � una normalizzazione vettoriale. Dato un vettore $x = [x_1, x_2, ..., x_n]$, � possibile calcolare il valore della norma L2 con la seguente formula:
	\begin{equation}
		|x| = \sqrt{\sum_{k=1}^{n} x_k^2}
	\end{equation}
	Infine, per normalizzarlo, bisogna dividere ogni valore di x con quello ottenuto dalla formula di norma L2.
	\\
	In uno spazio vettoriale occorre calcolare il valore di norma L2 per ogni riga o colonna della matrice, a seconda della scelta e dividere ogni elemento della riga o della colonna per quel valore.
\end{itemize}

Ai fini della sperimentazione, alla luce di quanto dichiarato dagli autori di \cite{norm}, si � deciso di utilizzare come tecnica la L2 per ogni riga. Infatti gli stessi autori hanno affermato che, dopo una serie di tentativi, la norma L2 fornisce risultati di normalizzazione superiori se applicata per ogni riga della matrice.

\section{Web page Clustering}
L'operazione di Clustering delle pagine di un sito Web, generalmente, pu� avvenire sfruttando la struttura connessa del sito o trattando le singole pagine come documenti. Nel primo caso, vengono applicate tecniche e metodologie derivanti dalla teoria dei grafi per partizionare il grafo e raggruppare le pagine simili. Nel secondo caso, invece, viene analizzato il contenuto testuale visivo della pagina, ovvero tutto quello che l'utente pu� percepire durante la navigazione sulle pagine di un sito Web.
\\\\
Gli hyperlink tra le pagine vengono usati per organizzare i contenuti, puntando a pagine differenti. Il contenuto testuale, data l'ambiguit� del linguaggio naturale, pu� fornire indizi sbagliati e considerare diverse pagine correlate solo per una differente distribuzione dei termini.
\\\\
In questa tesi si � scelto di utilizzare algoritmi di Clustering per raggruppare le pagine di un sito Web in base alla loro rappresentazione vettoriale. I vettori utilizzati per l'operazione di Clustering sono:
\begin{itemize}
	\item \textbf{Vettori dei link}. Sono state utilizzate tre tipologie di Word2Vec: Skip-Gram che considera solo il contesto sinistro ottimizzato (utilizzando il valore della b) che apprende da sequenze generate da Random Walk partendo dalla homepage; Skip-Gram che considera solo il contesto sinistro non ottimizzato (ignorando la b) che apprende da sequenze generate da Random Walk partendo dalla homepage; Skip-Gram che considera il contesto sia destro sia sinistro che apprende da sequenze generate da Random Walk standard.
	\\
	E' stato inoltre applicato sul grafo del sito Web LINE (Sezione \ref{line}), un altro algoritmo che produce gli embedding dei nodi in base alla loro prossimit� con gli altri. Questo algoritmo � stato applicato nelle sue due varianti: prossimit� di primo e secondo ordine.
	%\\
	%La differenza tra Word2Vec e LINE sta nel fatto che il primo considera relazioni indirette poich� l'apprendimento dipende dalla lunghezza dei passi generati dal Random Walker e dalla dimensione della finestra di contesto; il secondo, invece, analizza relazioni dirette o al pi� relazioni tra nodi fratelli: rispettivamente la prossimit� di primo e di secondo ordine.
	\\
	I vettori dei link sono stati normalizzati utilizzando L2 per ogni riga.
	\item \textbf{Vettori dei contenuti}. Le pagine del sito Web sono state opportunamente preprocessate per essere trattate come documenti. Sono stati rimossi i tag HTML, i caratteri di escape e non alfanumerici e le parole troppo frequenti ($> 90\%$) e poco frequenti ($< 5\%$).
	\\
	Successivamente, sono stati applicati Doc2Vec (in Sezione \ref{doc2vec}) e TF-IDF (in Sezione \ref{tfidf}) sulla pagina Web preprocessata. I vettori dei contenuti sono stati normalizzati utilizzando L2 per ogni riga.
	\item \textbf{Vettori dei link-contenuti}. Questa tipologia di vettori � stata prodotta andando a concatenare i vettori dei link e quelli dei contenuti aventi stesso codice identificativo dell'URL. Il risultato sar� un nuovo spazio vettoriale i cui vettori avranno dimensione $n + m$, dove $n$ � la lunghezza del vettore dei link ed $m$ � la lunghezza del vettore dei contenuti. Prima della concatenazione, i vettori di link e di contenuto sono stati normalizzati utilizzando L2 per ogni riga.
\end{itemize}

\section{Esempio di dataset}
Questa sezione ha il compito di spiegare come � stato strutturato il dataset per la sperimentazione. 
I file sono stati generati dal Web Crawling e dai Random Walks, usati per effettuare la sperimentazione descritta in questa tesi.
\\
Per spiegare il dataset � stato utilizzato come esempio il sito del dipartimento di informatica di Stanford, CA: \texttt{cs.stanford.edu}.

\paragraph{seedsMap.txt}
Questo file contiene le associazioni tra gli URL e il codice identificativo, usato per ridurre i tempi di elaborazione e spazio di archiviazione.
\\\\
\texttt{
cs.stanford.edu/csdcf/policies	73\\
cs.stanford.edu/admissions/reapplying	52\\
cs.stanford.edu/people/eaf/wordpress/videos	247\\
...
}

\paragraph{vertex.txt}
Contiene il contenuto testuale di ogni pagina esplorata. Ogni riga � formata dal codice identificativo di un URL ed il relativo contenuto.
\\\\
\texttt{
3	skip to skip to content skip to navigation webauth login ...\\
121	webauth error webauth error an error has occurred error ...\\
90	skip to content skip to navigation webauth login sunetid ...\\
...
}

\paragraph{edges.txt}
Contiene gli archi tra i nodi del grafo del sito web, i quali non sono altro che i collegamenti tra le pagine. Questo file viene usato per la generazione delle sequenze dall'algoritmo dei Random Walks.
\\\\
\texttt{
3	69\\
3	101\\
3	88\\
...
}

\paragraph{sequenceIDs.txt}
Contiene le sequenze generate da un Random Walker. Nel file vengono riportati i passi generati dall'algoritmo di Random Walk partendo da un nodo casuale del grafo del sito Web.
\\\\
\texttt{
109 33 106 89 10 108 57 91 8 51\\
80 42 17 95 66 109 109 78 44 22\\
206 280\\
...
}

\paragraph{sequenceIDsFromHomepage.txt}
Contiene le sequenze generate da un Random Walker. Nel file vengono riportati i passi generati dall'algoritmo di Random Walk specificando il nodo di partenza (i.e. la homepage) di ogni sequenza.
\\\\
\texttt{
3 37 72\\
3 12 48 47 22 64 74 48 36 8\\
3 6 26 12 57 63 95 109 87 71\\
...
}

\paragraph{embeddings\_with\_b.txt, embeddings\_no\_b.txt, embeddings\_normal.txt, embeddings\_line\_first.txt,\\ embeddings\_line\_second.txt, embeddings\_doc2vec.txt}
\texttt{\\embeddings\_with\_b.txt} ed \texttt{embeddings\_no\_b.txt} contengono gli embedding degli URL appresi da sequenze generate da Random Walk con partenza da homepage, che sono stati generati dal modello di apprendimento Skip-Gram, rispettivamente ottimizzato e non, che considera solo il contesto sinistro, data una parola.
\\
\texttt{embeddings\_normal.txt} contiene gli embedding degli URL appresi da sequenze prodotte da Random Walk standard, che sono stati generati dal modello di apprendimento Skip-Gram che considera il contesto destro e sinistro, data una parola.
\\
\texttt{embeddings\_line\_first.txt} ed \texttt{embeddings\_line\_second.txt} contengono gli embedding degli URL appresi dal file degli archi del grafo del sito Web. E' stato utilizzato LINE rispettivamente con prossimit� di primo e secondo ordine.
\\
\texttt{embeddings\_doc2vec.txt} contiene gli embedding del contenuto delle pagine del grafo del sito Web, dove ognuno dei vettori � associato ad un codice identificativo dell'URL.
\\\\
\texttt{
80 0.030692825093865395 0.09835819154977798 ...\\
44 -0.06424273550510406 -0.0433584563434124 ...\\
69 -0.1409958302974701 -0.013164487667381763 ...\\
...
}

\paragraph{groundTruth.csv}
Contiene la tavola di verit�, in cui ad ogni URL � associata una etichetta che indica il cluster di appartenenza. Questo file viene utilizzato per misurare la bont� dell'algoritmo di Clustering ed � stato usato nella sperimentazione. Nella tavola di verit� � presente come etichetta -1: se un URL presenta questo valore, significa che la pagina non � stata assegnata ad alcun raggruppamento.
\\\\
\texttt{
cs.stanford.edu/ip	-1\\
cs.stanford.edu/about/contact-us	1\\
cs.stanford.edu/academics	2\\
cs.stanford.edu/academics/phd	2\\
cs.stanford.edu/admissions	3\\
cs.stanford.edu/computing-guide	4\\
...
}

\chapter{Sperimentazione}
\label{cap:capitolo4}
\documentclass[a4paper,12pt,oneside]{book}
\usepackage[usenames]{color} %usato per il colore
\usepackage{amssymb} %maths
\usepackage{amsmath} %maths
\usepackage[latin1]{inputenc}
\usepackage[pdftex]{graphicx} 
\usepackage{filecontents}
\usepackage[italian]{babel}
\usepackage{setspace}
\usepackage[flushleft]{threeparttable}


\graphicspath{ {./../Immagini/} }

\begin{document}

\addtolength{\oddsidemargin}{+1,0cm}
\addtolength{\evensidemargin}{+1,0cm}
\onehalfspacing

\tableofcontents
\listoftables
\listoffigures
\newpage

In questo capitolo verranno descritte le modalit� di esecuzione della sperimentazione.
\\\\
L'obiettivo di questo capitolo � quello di valutare l'efficacia dell'approccio realizzato per poter comprendere se la modifica del modello di Word2Vec Skip-Gram pu� portare ad un miglioramento effettivo porta ad un effettivo miglioramento del processo di Clustering delle pagine di un sito Web.
\\
Si � cercato di capire, inoltre, se e come i dati strutturati, di cui un sito Web si compone, possano essere combinati con dati non strutturati, come quelli testuali, al fine di migliorare il processo di Clustering e valutare in dettaglio le cause di un eventuale successo o insuccesso delle tecniche utilizzate. A tal fine si confronteranno le performance degli algoritmi di Clustering basati su rappresentazione vettoriale contenente informazioni sulla struttura del grafo stesso.
\\\\
Si descrivono in seguito i dataset e le configurazioni degli algoritmi di Clustering su cui sono realizzate le sperimentazioni e le metriche utilizzate per valutare la qualit� dei cluster estratti.

\section{Dataset}
\label{dataset}
La sperimentazione � stata effettuata sfruttando i dati provenienti da siti Web appartenenti ad importanti dipartimenti di Computer Science: \textbf{Illinois} (\texttt{cs.illinois.edu}), \textbf{Oxford} (\texttt{cs.ox.ac.uk}), \textbf{Priceton} (\texttt{cs.priceton.edu}) e \textbf{Stanford} (\texttt{cs.stanford.edu}). La motivazione di questa scelta � legata al fatto che le nostre competenze, per assegnare ad ogni pagina del sito una etichetta, appartengono a questo dominio applicativo. E' stato necessario per creare una ground truth per la valutazione dei risultati del Clustering.
\\\\
Per ogni sito � stato lanciato il Crawling per generare sequenze di 100.000, 500.000 e 1.000.000 con profondit� di 10, 15, 20. Una volta prodotte sono state successivamente apprese da Skip-Gram ottimizzato che considera solo il contesto sinistro, Skip-Gram non ottimizzato che considera solo il contesto sinistro e Skip-Gram che considera contesto destro e sinistro con dimensione della finestra di apprendimento di 2, 3, 5, 7.

\begin{table}[h]
	\centering
	\caption{Descrizione dei siti Web}
	\label{websitedescr}
	\begin{tabular}{|l|l|l|l|l|}
		\hline
		Sito & Num. pagine & Num. archi & Num. archi con Liste Web & Num. Cluster \\ \hline
		Illinois & 563 & 9415 & 5330 & 10 \\ \hline
		Oxford & 3480 & 44526 & 35148 & 19 \\ \hline
		Priceton & 3132 & 122493 & 104585 & 16 \\ \hline
		Stanford & 167 & 12372 & 30087 & 10 \\ \hline
	\end{tabular}
\end{table}

In Tabella \ref{websitedescr} vengono riportate le dimensioni di ogni sito, una volta lanciato il Crawler. In particolare, per analizzare correttamente il contributo fornito dall'applicazione delle Liste Web nel processo di Clustering, sono state confrontate le pagine estratte sia dal Crawler che usa le Liste Web, sia da quello tradizionale (prima colonna). Inoltre, � stata riportata la dimensione della collezione degli archi ottenuta dal Crawler tradizionale (seconda colonna) e da quello che usa le Liste Web (terza colonna). Nell'ultima colonna sono stati inseriti il numero dei Cluster identificati manualmente dagli esperti, durante la generazione della ground truth.
\\\\
Per ogni sito Web, in pi�, sono stati estratti due differenti grafi:
\begin{itemize}
	\item \textbf{NoConstraint}. Il grafo Web $G_{nc} = (V, E)$ rappresenta fedelmente il sito. In particolare $V$ rappresenta l'insieme delle pagine appartenenti al sito ed $E$ � l'insieme di tutte le coppie $(i, j)$ per i quali la pagina con URL $i$ contiene un hyperlink alla pagina con URL $j$.
	\item \textbf{ListConstraint}. Il grafo Web $G_{lc} = (V, E)$ � filtrato, eliminando tutti gli archi $(i, j) \in E$ per i quali l'URL $j$ non � contenuto in nessuna lista Web nella pagina con URL $i$.
\end{itemize}

\section{Metriche}
Valutare le prestazioni di un algoritmo di Clustering non � semplice. Queste non dovrebbero considerare gli specifici valori delle etichette assegnate dall'algoritmo, piuttosto dovrebbero verificare se il raggruppamento generato dall'algoritmo definisce una separazione dei dati simile a quello fornito nella ground truth, ovvero il vero valore delle etichette. Oppure si potrebbe pensare di usare una qualche assunzione, come per esempio che elementi dello stesso raggruppamento siano pi� simili rispetto a quelli di Cluster differenti, usando una determinata funzione di similarit�.
\\
Per la sperimentazione si � scelto di utilizzare le metriche elencate in seguito per valutare l'efficacia dell'approccio proposto.

\begin{itemize}
	\item \textbf{Omogeneit�} \cite{vmeasure}. Ogni Cluster dovrebbe contenere elementi che sono membri di una stessa classe. Questa metrica viene calcolata dall'entropia condizionata della distribuzione di classe, dato il Cluster. Formalmente, l'omogeneit� viene definita come:
	\begin{equation}
		h = 1 - \frac{H(C|K)}{H(C)}
	\end{equation}
	dove $H(C|K)$ � l'entropia condizionata delle classi date le assegnazioni dei Cluster e $H(C)$ � l'entropia delle classi, ossia:
	\begin{equation}
	\label{hck}
		H(C|K) = - \sum_{k=1}^{|K|} \sum_{c=1}^{|C|} \frac{n_{ck}}{N} \log \frac{n_{ck}}{n_k}
	\end{equation}
	\begin{equation}
		H(C) = - \sum_{c=1}^{|C|} \frac{n_c}{N} \log \frac{n_c}{N}
	\end{equation}
	con $N$ il numero totale delle pagine, $n_c$ e $n_k$ il numero delle pagine appartenenti alla classe $c$ o al Cluster $k$, $n_ck$ il numero delle pagine della classe $c$ assegnate al cluster $k$.
	\\
	Il valore di questa metrica ha come intervallo $[0, 1]$. In una situazione ideale, l'Omogeneit� assume valore 1.
	
	\item \textbf{Completezza} \cite{vmeasure}. Tutti gli elementi che sono membri di una data classe dovrebbero appartenere ad uno stesso Cluster. Questa metrica � simmetrica all'omogeneit� ed � calcolata dall'entropia condizionata della distribuzione degli assegnamenti di ogni classe ad un dato Cluster. Formalmente, la Completezza � definita come segue:
	\begin{equation}
	c = 1 - \frac{H(C|K)}{H(K)}
	\end{equation}
	dove $H(C|K)$ � l'entropia condizionata delle classi date le assegnazioni dei Cluster, ed � stata definita nell'equazione \ref{hck}; $H(K)$ � l'entropia dei cluster ed � cos� formulata:
	\begin{equation}
		H(K) = - \sum_{k=1}^{|K|} \frac{n_k}{N} \log \frac{n_k}{N}
	\end{equation}
	con $N$ � il numero totale delle pagine ed $n_k$ � il numero delle pagine appartenenti al Cluster K.
	\\
	Come l'Omogeneit�, anche questa metrica ha come intervallo $[0, 1]$. In una soluzione perfetta, ogni distribuzione di classe dovrebbe convergere in un Cluster, avendo come Completezza 1.
	
	\item \textbf{V-Measure} \cite{vmeasure}. E' la media armonica tra l'Omogeneit� e la Completezza ed � formalmente descritta come segue:
	\begin{equation}
		v = 2 \cdot \frac{h \cdot c}{h + c}
	\end{equation}
	dove $h$ � il valore dell'Omogeneit� e $c$ quello della Completezza.
	
	\item \textbf{Adjusted Mutual Information (AMI)}. Questa metrica � una variazione della \textit{Mutual Information} (MI), ovvero una funzione che misura la corrispondenza delle due informazioni, ignorando le permutazioni. La MI viene definita come:
	\begin{equation}
		MI = \sum_{i \in K} \sum_{j \in C} \log \frac{P(i, j)}{P(i) P(j)}
	\end{equation}
	dove $C$ � l'insieme delle classi reali, $K$ � l'insieme dei Cluster appresi, $P(i, j)$ � la probabilit� che un elemento appartenente sia alla classe reale $i$ che a quella appresa $j$, $P(i)$ � la probabilit� a priori che un elemento appartenga alla classe $i$.
	\\
	La MI � generalmente pi� alta per due Clustering aventi un pi� grande numero di Cluster, non considerando il fatto che in realt� ci sono pi� informazioni condivise. La AMI rappresenta un adeguamento della MI per superare questa limitazione.
	\\
	Il valore di questa metrica ha come intervallo $[0, 1]$. In una situazione ideale, l'AMI assume valore 1.
	
	\item \textbf{Adjusted Random Index (ARI)} \cite{partitions}. Nota la ground truth e gli assegnamenti restituiti da un algoritmo di Clustering, questa metrica calcola la similarit� considerando tutte le coppie dei campioni e contando quelle che sono state assegnate nello stesso o in differenti Cluster, sia dalla tabella di verit� che dal processo di Clustering reale. In parole povere, l'ARI misura l'accuratezza del processo di Clustering, ossia la percentuale di coppie di oggetti per i quali la ground truth e l'algoritmo concordano sull'assegnazione. Come la AMI, anche qui si ignorano le permutazioni tra i due insiemi.
	\\
	Questa metrica dipende fortemente dal \textit{Random Index} (RI), che misura anch'essa la similarit� tra due processi di Clustering. Formalmente viene definita come:
	\begin{equation}
		RI = \frac{a + b}{\binom{n}{2}}
	\end{equation}
	dove $a$ � il numero di coppie di elementi che si trovano nella stessa classe e nell'insieme delle etichette predette, $b$ � il numero di coppie di elementi che appartengono a differenti classi e nell'insieme delle etichette predette, $\binom{n}{2}$ � il numero totale di tutte le possibili coppie nel dataset.
	Si � preferito utilizzare l'ARI poich� il RI, nel caso di Cluster computati in maniera casuale, non assume valori costanti (per esempio 0). Si garantisce, cos�, che Cluster generati in modo casuale abbiano come valore di ARI prossimi allo 0.
	\\
	Il valore di questa metrica ha come intervallo $[0, 1]$. In una situazione ideale, l'AMI assume valore 1, ovvero quando le etichette che si trovano nella ground truth e quelle predette sono identiche.
	
	\item \textbf{Silhouette}. Misura la forma di ogni cluster. Misura quanto simile � un elemento al Cluster a cui � assegnato (coesione) comparato con gli altri insiemi (separazione). Formalmente, questa metrica viene definita come:
	\begin{equation}
		s = \frac{b - a}{max(a, b)}
	\end{equation}
	dove $a$ � la distanza media tra un elemento e tutti gli altri della stessa classe, $b$ � la distanza media tra un elemento e tutti gli altri nella classe pi� vicina.
	\\
	Il valore di questa metrica, a differenza delle precedenti, ha come intervallo $[-1, 1]$. Un valore intorno allo 0 indica Cluster sovrapposti; -1 per cluster non definiti; 1 per elementi che sono altamente coesi con gli altri dello stesso raggruppamento ma bassamente coesi per quelli appartenenti a Cluster vicini.
\end{itemize}

\section{Configurazioni}
Si descrivono di seguito le configurazioni utilizzate per la sperimentazione in questa tesi.

\subsection{Testo}
Sono state applicate tecniche di Text Mining per il Clustering basato sul contenuto testuale. Con questa metodologia viene considerato solo l'informazione estratta dal testo, assumendo che i termini all'interno del sito Web siano indipendenti l'uno dall'altro, cos� come i documenti. Vengono ignorate, quindi, le relazioni interdipendenti tra questi. Il Web si discosta dall'analisi classica di testi per l'esistenza di relazioni tra le pagine. Tuttavia l'analisi testuale rimane una tecnica importante.
\\\\
Nella fase di sperimentazione sono stati utilizzati due algoritmi per estrarre informazioni dal testo: \textit{Doc2Vec} (esaminato in Sezione \ref{doc2vec}) e \textit{tf-idf} (esaminato in Sezione \ref{tfidf}).
\\\\
I parametri utilizzati in Doc2Vec sono stati:
\begin{itemize}
	\item \textbf{minWordFrequency}: questo valore rappresenta il numero di volte minime per considerare un determinato termine. E' stato impostato ad 1.
	\item \textbf{layerSize}: indica la dimensione dei vettori in output. E' stato impostato a 100.
	\item \textbf{windowSize}: E' la dimensione della finestra di contesto, utile per svolgere il training sulle parole del testo. E' stata impostata a 5.
\end{itemize}
I parametri, invece, per creare la matrice con tf-idf sono stati:
\begin{itemize}
	\item \textbf{max-df}: questo valore rappresenta la massima frequenza, all'interno dei documenti, che un termine pu� avere per essere utilizzato nella matrice tf-idf. Se un termine appare molte volte nel corpus, molto probabilmente avr� poco significato. E' stato impostato a 0.9.
	\item \textbf{min-df}: indica il numero minimo di documenti in cui un termine dovr� apparire per essere considerato. E' stato impostato a 0.05.
	\item \textbf{ngram-range}: vengono presi in considerazioni gli n-grammi di lunghezza compresa nell'intervallo specificato in questo parametro. Nello specifico, questo parametro � una tupla (min\_n, max\_n) che definisce il confine minimo e massimo dell'intervallo di valori n-esimi per differenti n-gram per essere estratti. Tutti i valori di n che si trovano nell'intervallo $min\_n <= n <= max\_n$ vengono considerati. Un n-gramma � una sottosequenza di n elementi di un'altra. E' stato impostato a (1, 2).
\end{itemize}
Successivamente, la dimensione dei vettori della matrice tf-idf � stata ridotta troncandoli. La dimensione dei vettori � stata ridotta a 100.
\\\\
I valori dei parametri sopra citati sono stati utilizzati per tutti i dataset della sperimentazione.

\subsection{Random Walk}
Considerando i Random Walk come frasi, � possibile applicare gli algoritmi di Word Embedding per raggruppare pagine di un sito Web sulla base del contesto in cui appaiono, ovvero le pagine che pi� verosimilmente appariranno insieme nelle sequenze.
\\
Queste frasi sono state generate da grafi di siti Web a liste di costrizione e senza costrizione (esaminate in sezione \ref{dataset}). Sono stati prodotti Random Walk che partono dalla homepage e quelli standard, che sono stati rispettivamente appresi dalla versione modificata di Skip-Gram (ottimizzato e non, che considera solo il contesto sinistro) e da Skip-Gram tradizionale. Per entrambi i casi, sono state create un numero di frasi di 100.000, 500.000 e 1.000.000 e la loro lunghezza massima di 10, 15, 20.
\\\\
Per tutte le versioni di Word2Vec Skip-Gram, sono stati utilizzati i seguenti paragrafi:
\begin{itemize}
	\item \textbf{minWordFrequency}: questo valore rappresenta il numero di volte minime per considerare un determinato termine. E' stato impostato ad 1.
	\item \textbf{layerSize}: indica la dimensione dei vettori in output. E' stato impostato a 100.
	\item \textbf{windowSize}: E' la dimensione della finestra di contesto, utile per svolgere il training sulle parole del testo. E' stata impostata a 2, 3, 5, 7.
	\item \textbf{iterate}: numero di iterazioni sulle frasi da apprendere. E' stato impostato a 50 per Random Walk di 100.000 frasi; 10 per Random Walk di 500.000 frasi; 3 per Random Walk di 1.000.000 di frasi.
\end{itemize}
I valori dei parametri sopra citati sono stati utilizzati per tutti i dataset della sperimentazione.

\subsection{Combinato}
Sono state combinate le informazioni riguardanti la struttura del sito Web e il contenuto testuale delle pagine. Le combinazioni prodotte per la sperimentazione sono di due tipi:
\begin{itemize}
	\item \textbf{Combined List Constraint}. Sono stati combinati i vettori prodotti dalla struttura ad hyperlink del sito, usando il Crawler che sfrutta le Liste Web, e quelli prodotti dal contenuto delle pagine. Sono stati prodotti vettori aventi come dimensione 200.
	\item \textbf{Combined No Constraint}. Come nella configurazione precedente, sono stati combinati testo e struttura ad hyperlink del sito Web in un singolo spazio vettoriale, avente come dimensione 200. In questo caso, � stato utilizzato il Crawler tradizionale.
\end{itemize}

\subsection{Archi}
Si � voluto confrontare il modello Skip-Gram tradizionale con un nuovo algoritmo di apprendimento, chiamato LINE (esaminato in Sezione \ref{line}). A differenza di Skip-Gram, LINE considera solo relazioni dirette (prossimit� di primo ordine) o al pi� relazioni tra nodi fratelli (prossimit� di secondo ordine). Richiede, per questo, il file contenente gli archi del sito Web, che viene generato dal Crawler.
\\\\
I parametri per utilizzare LINE di primo e secondo ordine sono:
\begin{itemize}
	\item \textbf{-train}: path del file contenente gli archi del grafo Web, ovvero \texttt{edges.txt}.
	\item \textbf{-output}: path del file contenente gli embeddings prodotti dall'algoritmo.
	\item \textbf{-order}: tipo di prossimit� per produrre gli embeddings, ovvero 1 per primo ordine e 2 per secondo ordine.
\end{itemize}

\subsection{Algoritmi di Clustering}
Una volta prodotti gli embeddings dai rispettivi algoritmi, sono stati normalizzati secondo la norma L2 (esaminata in Sezione \ref{scaling}) ed � stato effettuato il Clustering delle pagine del sito Web. Da questo processo sono stati esclusi tutte le pagine che presentavano nella ground truth etichette pari a -1, poich� tale valore significa che l'esperto che ha creato la tabella di verit� non � riuscito ad assegnare a nessun raggruppamento la pagina in questione.
\\\\
In K-Means (approfondito in Sezione \ref{clusteringexperimentation}), il parametro utilizzato nella sperimentazione � stato \textbf{n\_clusters}, ovvero il numero di Cluster totale da generare. Per rendere il sistema pi� dinamico, si � scelto di settare questo parametro al numero delle etichette presente nella ground truth.
\\\\
In HDBScan (approfondito in Sezione \ref{clusteringexperimentation}), il parametro utilizzato nella sperimentazione � stato \textbf{min\_cluster\_size}, ovvero il numero minimo di elementi che un raggruppamento deve contenere per essere considerato come Cluster. E' stato impostato a 5.

\section{Analisi dei risultati}
Di seguito si confrontano i risultati degli algoritmi di Clustering basati sulle configurazioni sopra citate, ovvero: \textit{Random Walk}, \textit{Testo} e \textit{Combinato}. Sono stati riportati solo i risultati di Skip-Gram ottimizzato che considera solo il contesto sinistro (\textit{leftSkipgram}) e Skip-Gram tradizionale (\textit{normalSkipgram}) poich�, nella maggior parte dei casi, leftSkipgram forniva risultati migliori rispetto alla versione modificata non ottimizzata.
\\\\
Per motivi di praticit�, nella configurazione dei Random Walk � stata utilizzata la dicitura $[db, window, depth]$, dove $db$ rappresenta il numero di frasi da far apprendere all'algoritmo, $window$ la lunghezza della finestra di contesto e $depth$ la lunghezza massima della singola frase.
\\\\
Inoltre, per motivi di spazio, si � scelta la dicitura \textit{LINE-1} e \textit{LINE-2} per riferirsi rispettivamente a LINE con primo ordine di prossimit� e LINE con secondo ordine di prossimit�.

\subsection{cs.illinois.edu}

\subsubsection{Random Walk}

\begin{table}[h]
	\centering
	\caption{Risultati di KMeans con leftSkipgram con liste di costrizione nella configurazione Random Walk in Illinois}
	\label{leftskipgramrwlckmeansillinois}
	\begin{tabular}{|l|l|l|l|l|l|l|l|l|l|l|l|}
		\hline
		\textbf{DB size} & \textbf{Window} & \textbf{RW len.} & \textbf{AMI} & \textbf{ARI} & \textbf{Com} & \textbf{Hom} & \textbf{Silh} & \textbf{V-M} \\ \hline
		\textbf{100K} & \textbf{2} & \textbf{10} & 0.26 & 0.12 & 0.28 & 0.37 & -0.03 & 0.32 \\ \hline
		\textbf{100K} & \textbf{3} & \textbf{10} & 0.35 & 0.24 & 0.37 & 0.45 & -0.03 & 0.41 \\ \hline
		\textbf{100K} & \textbf{5} & \textbf{10} & 0.60 & \textbf{0.62} & 0.62 & 0.65 & -0.04 & 0.63 \\ \hline
		\textbf{100K} & \textbf{7} & \textbf{10} & 0.58 & 0.61 & 0.59 & 0.65 & -0.06 & 0.62 \\ \hline
		\textbf{100K} & \textbf{2} & \textbf{15} & 0.26 & 0.12 & 0.28 & 0.35 & -0.02 & 0.31 \\ \hline
		\textbf{100K} & \textbf{3} & \textbf{15} & 0.46 & 0.30 & 0.48 & 0.63 & -0.04 & 0.54 \\ \hline
		\textbf{100K} & \textbf{5} & \textbf{15} & 0.56 & 0.52 & 0.58 & 0.69 & -0.05 & 0.63 \\ \hline
		\textbf{100K} & \textbf{7} & \textbf{15} & 0.54 & 0.49 & 0.55 & 0.68 & -0.06 & 0.61 \\ \hline
		\textbf{100K} & \textbf{2} & \textbf{20} & 0.26 & 0.11 & 0.28 & 0.37 & -0.02 & 0.32 \\ \hline
		\textbf{100K} & \textbf{3} & \textbf{20} & 0.38 & 0.24 & 0.40 & 0.51 & -0.04 & 0.45 \\ \hline
		\textbf{100K} & \textbf{5} & \textbf{20} & 0.43 & 0.23 & 0.45 & 0.57 & -0.06 & 0.50 \\ \hline
		\textbf{100K} & \textbf{7} & \textbf{20} & 0.45 & 0.30 & 0.47 & 0.56 & -0.04 & 0.51 \\ \hline
		\textbf{500K} & \textbf{2} & \textbf{10} & 0.23 & 0.09 & 0.26 & 0.34 & -0.02 & 0.29 \\ \hline
		\textbf{500K} & \textbf{3} & \textbf{10} & 0.53 & 0.35 & 0.54 & 0.71 & -0.03 & 0.62 \\ \hline
		\textbf{500K} & \textbf{5} & \textbf{10} & 0.61 & 0.48 & 0.62 & 0.77 & -0.05 & 0.68 \\ \hline
		\textbf{500K} & \textbf{7} & \textbf{10} & 0.47 & 0.29 & 0.49 & 0.61 & -0.04 & 0.54 \\ \hline
		\textbf{500K} & \textbf{2} & \textbf{15} & 0.24 & 0.10 & 0.26 & 0.35 & -0.02 & 0.30 \\ \hline
		\textbf{500K} & \textbf{3} & \textbf{15} & 0.59 & 0.41 & 0.60 & 0.80 & -0.04 & 0.69 \\ \hline
		\textbf{500K} & \textbf{5} & \textbf{15} & 0.59 & 0.41 & 0.60 & 0.80 & -0.04 & 0.69 \\ \hline
		\textbf{500K} & \textbf{7} & \textbf{15} & 0.50 & 0.29 & 0.51 & 0.69 & -0.03 & 0.59 \\ \hline
		\textbf{500K} & \textbf{2} & \textbf{20} & 0.20 & 0.09 & 0.23 & 0.30 & \textbf{-0.01} & 0.26 \\ \hline
		\textbf{500K} & \textbf{3} & \textbf{20} & 0.61 & 0.42 & \textbf{0.63} & 0.82 & -0.04 & 0.71 \\ \hline
		\textbf{500K} & \textbf{5} & \textbf{20} & 0.61 & 0.39 & \textbf{0.63} & 0.83 & -0.04 & 0.71 \\ \hline
		\textbf{500K} & \textbf{7} & \textbf{20} & \textbf{0.62} & 0.42 & \textbf{0.63} & 0.83 & -0.03 & \textbf{0.72} \\ \hline
		\textbf{1M} & \textbf{2} & \textbf{10} & 0.24 & 0.10 & 0.27 & 0.35 & -0.02 & 0.30 \\ \hline
		\textbf{1M} & \textbf{3} & \textbf{10} & 0.58 & 0.38 & 0.59 & 0.79 & -0.06 & 0.68 \\ \hline
		\textbf{1M} & \textbf{5} & \textbf{10} & 0.57 & 0.37 & 0.59 & 0.76 & -0.05 & 0.66 \\ \hline
		\textbf{1M} & \textbf{7} & \textbf{10} & 0.52 & 0.32 & 0.53 & 0.70 & -0.07 & 0.61 \\ \hline
		\textbf{1M} & \textbf{2} & \textbf{15} & 0.24 & 0.10 & 0.26 & 0.34 & -0.02 & 0.30 \\ \hline
		\textbf{1M} & \textbf{3} & \textbf{15} & 0.60 & 0.38 & 0.61 & 0.82 & -0.04 & 0.70 \\ \hline
		\textbf{1M} & \textbf{5} & \textbf{15} & 0.56 & 0.36 & 0.58 & 0.77 & -0.04 & 0.66 \\ \hline
		\textbf{1M} & \textbf{7} & \textbf{15} & 0.56 & 0.34 & 0.58 & 0.77 & -0.04 & 0.66 \\ \hline
		\textbf{1M} & \textbf{2} & \textbf{20} & 0.23 & 0.09 & 0.26 & 0.34 & -0.02 & 0.30 \\ \hline
		\textbf{1M} & \textbf{3} & \textbf{20} & 0.61 & 0.38 & 0.62 & 0.83 & -0.04 & 0.71 \\ \hline
		\textbf{1M} & \textbf{5} & \textbf{20} & 0.61 & 0.40 & 0.62 & 0.83 & -0.04 & 0.71 \\ \hline
		\textbf{1M} & \textbf{7} & \textbf{20} & 0.60 & 0.37 & 0.62 & \textbf{0.84} & -0.04 & 0.71 \\ \hline
	\end{tabular}
\end{table}

\begin{table}[h]
	\centering
	\caption{Risultati di HDBScan con leftSkipgram con liste di costrizione nella configurazione Random Walk in Illinois}
	\label{leftskipgramrwlchdbscanillinois}
	\begin{tabular}{|l|l|l|l|l|l|l|l|l|l|}
		\hline
		\textbf{DB size} & \textbf{Window} & \textbf{RW len.} & \textbf{AMI} & \textbf{ARI} & \textbf{Com} & \textbf{Hom} & \textbf{Silh} & \textbf{V-M} \\ \hline
		\textbf{100K} & \textbf{2} & \textbf{10} & 0.06 & 0.04 & 0.29 & 0.07 & -0.01 & 0.12 \\ \hline
		\textbf{100K} & \textbf{3} & \textbf{10} & 0.06 & 0.00 & 0.17 & 0.07 & -0.02 & 0.10 \\ \hline
		\textbf{100K} & \textbf{5} & \textbf{10} & 0.06 & 0.10 & 0.23 & 0.07 & 0.01 & 0.11 \\ \hline
		\textbf{100K} & \textbf{7} & \textbf{10} & 0.14 & 0.16 & 0.30 & 0.15 & -0.04 & 0.20 \\ \hline
		\textbf{100K} & \textbf{2} & \textbf{15} & 0.11 & 0.08 & 0.39 & 0.11 & -0.02 & 0.18 \\ \hline
		\textbf{100K} & \textbf{3} & \textbf{15} & 0.08 & 0.11 & 0.22 & 0.09 & 0.00 & 0.13 \\ \hline
		\textbf{100K} & \textbf{5} & \textbf{15} & 0.08 & 0.07 & 0.30 & 0.09 & -0.05 & 0.14 \\ \hline
		\textbf{100K} & \textbf{7} & \textbf{15} & 0.13 & 0.09 & 0.28 & 0.15 & -0.03 & 0.19 \\ \hline
		\textbf{100K} & \textbf{2} & \textbf{20} & 0.20 & 0.18 & \textbf{0.54} & 0.20 & -0.01 & 0.30 \\ \hline
		\textbf{100K} & \textbf{3} & \textbf{20} & 0.15 & 0.11 & 0.29 & 0.17 & -0.04 & 0.21 \\ \hline
		\textbf{100K} & \textbf{5} & \textbf{20} & 0.16 & 0.18 & 0.33 & 0.18 & -0.02 & 0.23 \\ \hline
		\textbf{100K} & \textbf{7} & \textbf{20} & 0.21 & \textbf{0.20} & 0.42 & 0.22 & -0.01 & 0.29 \\ \hline
		\textbf{500K} & \textbf{2} & \textbf{10} & 0.21 & 0.15 & 0.53 & 0.22 & -0.02 & 0.31 \\ \hline
		\textbf{500K} & \textbf{3} & \textbf{10} & 0.18 & \textbf{0.20} & 0.35 & 0.19 & -0.04 & 0.25 \\ \hline
		\textbf{500K} & \textbf{5} & \textbf{10} & 0.19 & 0.15 & 0.36 & 0.22 & -0.11 & 0.27 \\ \hline
		\textbf{500K} & \textbf{7} & \textbf{10} & 0.20 & 0.16 & 0.43 & 0.22 & -0.15 & 0.29 \\ \hline
		\textbf{500K} & \textbf{2} & \textbf{15} & 0.18 & 0.13 & 0.51 & 0.19 & -0.01 & 0.27 \\ \hline
		\textbf{500K} & \textbf{3} & \textbf{15} & 0.16 & 0.15 & 0.43 & 0.17 & \textbf{0.02} & 0.24 \\ \hline
		\textbf{500K} & \textbf{5} & \textbf{15} & 0.05 & 0.01 & 0.23 & 0.06 & 0.01 & 0.09 \\ \hline
		\textbf{500K} & \textbf{7} & \textbf{15} & 0.07 & 0.04 & 0.28 & 0.09 & -0.04 & 0.13 \\ \hline
		\textbf{500K} & \textbf{2} & \textbf{20} & 0.15 & 0.09 & 0.46 & 0.16 & -0.01 & 0.24 \\ \hline
		\textbf{500K} & \textbf{3} & \textbf{20} & 0.41 & 0.13 & 0.44 & 0.63 & -0.11 & 0.52 \\ \hline
		\textbf{500K} & \textbf{5} & \textbf{20} & 0.21 & \textbf{0.20} & 0.53 & 0.22 & -0.01 & 0.31 \\ \hline
		\textbf{500K} & \textbf{7} & \textbf{20} & 0.14 & 0.11 & 0.41 & 0.15 & -0.02 & 0.22 \\ \hline
		\textbf{1M} & \textbf{2} & \textbf{10} & 0.19 & 0.12 & 0.53 & 0.21 & -0.03 & 0.30 \\ \hline
		\textbf{1M} & \textbf{3} & \textbf{10} & 0.41 & 0.15 & 0.45 & 0.65 & -0.15 & 0.53 \\ \hline
		\textbf{1M} & \textbf{5} & \textbf{10} & 0.43 & 0.19 & 0.47 & 0.73 & -0.14 & 0.57 \\ \hline
		\textbf{1M} & \textbf{7} & \textbf{10} & 0.43 & 0.18 & 0.47 & 0.72 & -0.15 & 0.56 \\ \hline
		\textbf{1M} & \textbf{2} & \textbf{15} & 0.17 & 0.09 & 0.49 & 0.18 & -0.03 & 0.26 \\ \hline
		\textbf{1M} & \textbf{3} & \textbf{15} & 0.40 & 0.15 & 0.45 & 0.75 & -0.15 & 0.56 \\ \hline
		\textbf{1M} & \textbf{5} & \textbf{15} & \textbf{0.44} & 0.19 & 0.47 & \textbf{0.78} & -0.14 & \textbf{0.59} \\ \hline
		\textbf{1M} & \textbf{7} & \textbf{15} & 0.41 & 0.16 & 0.45 & \textbf{0.78} & -0.14 & 0.57 \\ \hline
		\textbf{1M} & \textbf{2} & \textbf{20} & 0.17 & 0.11 & 0.48 & 0.18 & -0.02 & 0.27 \\ \hline
		\textbf{1M} & \textbf{3} & \textbf{20} & 0.41 & 0.17 & 0.45 & 0.71 & -0.10 & 0.55 \\ \hline
		\textbf{1M} & \textbf{5} & \textbf{20} & 0.40 & 0.14 & 0.44 & 0.76 & -0.13 & 0.56 \\ \hline
		\textbf{1M} & \textbf{7} & \textbf{20} & 0.39 & 0.14 & 0.43 & 0.76 & -0.15 & 0.55 \\ \hline
	\end{tabular}
\end{table}

\begin{table}[h]
	\centering
	\caption{Risultati di KMeans con normalSkipgram con liste di costrizione nella configurazione Random Walk in Illinois}
	\label{normalskipgramrwlckmeansillinois}
	\begin{tabular}{|l|l|l|l|l|l|l|l|l|l|}
		\hline
		\textbf{DB size} & \textbf{Window} & \textbf{RW len.} & \textbf{AMI} & \textbf{ARI} & \textbf{Com} & \textbf{Hom} & \textbf{Silh} & \textbf{V-M} \\ \hline
		\textbf{100K} & \textbf{2} & \textbf{10} & 0.53 & 0.33 & 0.55 & 0.73 & -0.05 & 0.62 \\ \hline
		\textbf{100K} & \textbf{3} & \textbf{10} & 0.61 & \textbf{0.51} & 0.62 & 0.79 & -0.06 & 0.70 \\ \hline
		\textbf{100K} & \textbf{5} & \textbf{10} & 0.55 & 0.38 & 0.57 & 0.76 & -0.05 & 0.65 \\ \hline
		\textbf{100K} & \textbf{7} & \textbf{10} & 0.54 & 0.36 & 0.55 & 0.73 & -0.05 & 0.63 \\ \hline
		\textbf{100K} & \textbf{2} & \textbf{15} & 0.64 & 0.44 & 0.65 & 0.86 & -0.05 & 0.74 \\ \hline
		\textbf{100K} & \textbf{3} & \textbf{15} & 0.59 & 0.41 & 0.60 & 0.80 & -0.05 & 0.69 \\ \hline
		\textbf{100K} & \textbf{5} & \textbf{15} & 0.60 & 0.42 & 0.61 & 0.82 & -0.06 & 0.70 \\ \hline
		\textbf{100K} & \textbf{7} & \textbf{15} & 0.56 & 0.39 & 0.57 & 0.76 & -0.06 & 0.65 \\ \hline
		\textbf{100K} & \textbf{2} & \textbf{20} & 0.60 & 0.38 & 0.61 & 0.82 & -0.05 & 0.70 \\ \hline
		\textbf{100K} & \textbf{3} & \textbf{20} & 0.56 & 0.41 & 0.58 & 0.75 & -0.04 & 0.66 \\ \hline
		\textbf{100K} & \textbf{5} & \textbf{20} & 0.35 & 0.13 & 0.37 & 0.47 & -0.04 & 0.42 \\ \hline
		\textbf{100K} & \textbf{7} & \textbf{20} & 0.31 & 0.10 & 0.34 & 0.43 & -0.04 & 0.38 \\ \hline
		\textbf{500K} & \textbf{2} & \textbf{10} & 0.65 & 0.48 & \textbf{0.67} & 0.87 & -0.04 & 0.75 \\ \hline
		\textbf{500K} & \textbf{3} & \textbf{10} & 0.62 & 0.40 & 0.63 & 0.85 & \textbf{-0.03} & 0.72 \\ \hline
		\textbf{500K} & \textbf{5} & \textbf{10} & \textbf{0.66} & 0.46 & \textbf{0.67} & \textbf{0.89} & \textbf{-0.03} & \textbf{0.76} \\ \hline
		\textbf{500K} & \textbf{7} & \textbf{10} & 0.56 & 0.29 & 0.58 & 0.77 & \textbf{-0.03} & 0.66 \\ \hline
		\textbf{500K} & \textbf{2} & \textbf{15} & 0.60 & 0.36 & 0.62 & 0.83 & -0.05 & 0.71 \\ \hline
		\textbf{500K} & \textbf{3} & \textbf{15} & 0.58 & 0.33 & 0.59 & 0.80 & -0.04 & 0.68 \\ \hline
		\textbf{500K} & \textbf{5} & \textbf{15} & 0.61 & 0.38 & 0.62 & 0.83 & \textbf{-0.03} & 0.71 \\ \hline
		\textbf{500K} & \textbf{7} & \textbf{15} & 0.55 & 0.33 & 0.57 & 0.77 & \textbf{-0.03} & 0.65 \\ \hline
		\textbf{500K} & \textbf{2} & \textbf{20} & 0.65 & 0.43 & 0.66 & 0.86 & -0.04 & 0.75 \\ \hline
		\textbf{500K} & \textbf{3} & \textbf{20} & 0.58 & 0.36 & 0.59 & 0.80 & -0.05 & 0.68 \\ \hline
		\textbf{500K} & \textbf{5} & \textbf{20} & 0.58 & 0.34 & 0.59 & 0.79 & \textbf{-0.03} & 0.68 \\ \hline
		\textbf{500K} & \textbf{7} & \textbf{20} & 0.57 & 0.37 & 0.58 & 0.78 & \textbf{-0.03} & 0.67 \\ \hline
		\textbf{1M} & \textbf{2} & \textbf{10} & 0.61 & 0.37 & 0.62 & 0.83 & -0.05 & 0.71 \\ \hline
		\textbf{1M} & \textbf{3} & \textbf{10} & 0.63 & 0.40 & 0.64 & 0.86 & -0.05 & 0.73 \\ \hline
		\textbf{1M} & \textbf{5} & \textbf{10} & 0.62 & 0.40 & 0.64 & 0.85 & -0.04 & 0.73 \\ \hline
		\textbf{1M} & \textbf{7} & \textbf{10} & 0.61 & 0.37 & 0.62 & 0.84 & -0.04 & 0.71 \\ \hline
		\textbf{1M} & \textbf{2} & \textbf{15} & 0.61 & 0.38 & 0.62 & 0.84 & -0.05 & 0.71 \\ \hline
		\textbf{1M} & \textbf{3} & \textbf{15} & 0.60 & 0.36 & 0.61 & 0.83 & -0.04 & 0.70 \\ \hline
		\textbf{1M} & \textbf{5} & \textbf{15} & 0.60 & 0.37 & 0.61 & 0.82 & -0.04 & 0.70 \\ \hline
		\textbf{1M} & \textbf{7} & \textbf{15} & 0.58 & 0.36 & 0.59 & 0.80 & -0.04 & 0.68 \\ \hline
		\textbf{1M} & \textbf{2} & \textbf{20} & 0.58 & 0.35 & 0.60 & 0.80 & -0.04 & 0.68 \\ \hline
		\textbf{1M} & \textbf{3} & \textbf{20} & 0.64 & 0.42 & 0.65 & 0.85 & -0.05 & 0.74 \\ \hline
		\textbf{1M} & \textbf{5} & \textbf{20} & 0.61 & 0.39 & 0.62 & 0.82 & -0.04 & 0.70 \\ \hline
		\textbf{1M} & \textbf{7} & \textbf{20} & 0.57 & 0.36 & 0.58 & 0.77 & -0.04 & 0.67 \\ \hline
	\end{tabular}
\end{table}

\begin{table}[h]
	\centering
	\caption{Risultati di HDBScan con normalSkipgram con liste di costrizione nella configurazione Random Walk in Illinois}
	\label{leftskipgramrwlchdbscanillinois}
	\begin{tabular}{|l|l|l|l|l|l|l|l|l|l|}
		\hline
		\textbf{DB size} & \textbf{Window} & \textbf{RW len.} & \textbf{AMI} & \textbf{ARI} & \textbf{Com} & \textbf{Hom} & \textbf{Silh} & \textbf{V-M} \\ \hline
		\textbf{100K} & \textbf{2} & \textbf{10} & 0.40 & 0.15 & 0.45 & 0.57 & -0.12 & 0.50 \\ \hline
		\textbf{100K} & \textbf{3} & \textbf{10} & 0.39 & 0.12 & 0.44 & 0.54 & -0.12 & 0.48 \\ \hline
		\textbf{100K} & \textbf{5} & \textbf{10} & 0.39 & 0.11 & 0.43 & 0.50 & -0.12 & 0.46 \\ \hline
		\textbf{100K} & \textbf{7} & \textbf{10} & 0.38 & 0.08 & 0.42 & 0.49 & -0.11 & 0.45 \\ \hline
		\textbf{100K} & \textbf{2} & \textbf{15} & 0.39 & 0.09 & 0.43 & 0.59 & \textbf{-0.08} & 0.49 \\ \hline
		\textbf{100K} & \textbf{3} & \textbf{15} & 0.39 & 0.11 & 0.43 & 0.52 & \textbf{-0.08} & 0.47 \\ \hline
		\textbf{100K} & \textbf{5} & \textbf{15} & 0.36 & 0.09 & 0.41 & 0.48 & -0.11 & 0.44 \\ \hline
		\textbf{100K} & \textbf{7} & \textbf{15} & 0.37 & 0.12 & 0.41 & 0.48 & -0.10 & 0.44 \\ \hline
		\textbf{100K} & \textbf{2} & \textbf{20} & 0.36 & 0.09 & 0.41 & 0.60 & -0.17 & 0.49 \\ \hline
		\textbf{100K} & \textbf{3} & \textbf{20} & 0.35 & 0.06 & 0.40 & 0.53 & -0.15 & 0.46 \\ \hline
		\textbf{100K} & \textbf{5} & \textbf{20} & 0.13 & 0.03 & 0.24 & 0.16 & -0.13 & 0.20 \\ \hline
		\textbf{100K} & \textbf{7} & \textbf{20} & 0.19 & 0.06 & 0.30 & 0.21 & -0.09 & 0.25 \\ \hline
		\textbf{500K} & \textbf{2} & \textbf{10} & 0.40 & 0.15 & 0.45 & 0.77 & -0.11 & 0.56 \\ \hline
		\textbf{500K} & \textbf{3} & \textbf{10} & 0.41 & 0.14 & 0.45 & 0.73 & -0.10 & 0.55 \\ \hline
		\textbf{500K} & \textbf{5} & \textbf{10} & 0.39 & 0.11 & 0.43 & 0.67 & -0.09 & 0.53 \\ \hline
		\textbf{500K} & \textbf{7} & \textbf{10} & 0.41 & 0.14 & 0.45 & 0.69 & \textbf{-0.08} & 0.54 \\ \hline
		\textbf{500K} & \textbf{2} & \textbf{15} & 0.40 & 0.15 & 0.45 & 0.76 & -0.11 & 0.56 \\ \hline
		\textbf{500K} & \textbf{3} & \textbf{15} & 0.41 & 0.15 & 0.46 & 0.74 & -0.09 & 0.56 \\ \hline
		\textbf{500K} & \textbf{5} & \textbf{15} & 0.41 & 0.15 & 0.46 & 0.73 & \textbf{-0.08} & 0.56 \\ \hline
		\textbf{500K} & \textbf{7} & \textbf{15} & 0.41 & 0.16 & 0.45 & 0.72 & \textbf{-0.08} & 0.55 \\ \hline
		\textbf{500K} & \textbf{2} & \textbf{20} & 0.41 & 0.15 & 0.45 & 0.76 & -0.12 & 0.56 \\ \hline
		\textbf{500K} & \textbf{3} & \textbf{20} & \textbf{0.43} & 0.16 & \textbf{0.47} & 0.74 & -0.10 & 0.57 \\ \hline
		\textbf{500K} & \textbf{5} & \textbf{20} & 0.42 & 0.16 & 0.46 & 0.74 & -0.09 & 0.57 \\ \hline
		\textbf{500K} & \textbf{7} & \textbf{20} & \textbf{0.43} & \textbf{0.17} & \textbf{0.47} & 0.74 & \textbf{-0.08} & 0.58 \\ \hline
		\textbf{1M} & \textbf{2} & \textbf{10} & 0.41 & 0.16 & 0.45 & \textbf{0.79} & -0.12 & 0.58 \\ \hline
		\textbf{1M} & \textbf{3} & \textbf{10} & 0.41 & 0.15 & 0.45 & 0.77 & -0.13 & 0.57 \\ \hline
		\textbf{1M} & \textbf{5} & \textbf{10} & 0.42 & \textbf{0.17} & 0.46 & \textbf{0.79} & -0.10 & \textbf{0.59} \\ \hline
		\textbf{1M} & \textbf{7} & \textbf{10} & 0.41 & 0.15 & 0.45 & 0.77 & -0.11 & 0.57 \\ \hline
		\textbf{1M} & \textbf{2} & \textbf{15} & 0.39 & 0.14 & 0.44 & 0.74 & -0.15 & 0.55 \\ \hline
		\textbf{1M} & \textbf{3} & \textbf{15} & 0.40 & 0.15 & 0.44 & 0.76 & -0.12 & 0.56 \\ \hline
		\textbf{1M} & \textbf{5} & \textbf{15} & 0.39 & 0.14 & 0.44 & 0.77 & -0.13 & 0.56 \\ \hline
		\textbf{1M} & \textbf{7} & \textbf{15} & 0.38 & 0.14 & 0.43 & 0.75 & -0.13 & 0.55 \\ \hline
		\textbf{1M} & \textbf{2} & \textbf{20} & 0.41 & 0.16 & 0.45 & 0.77 & -0.14 & 0.57 \\ \hline
		\textbf{1M} & \textbf{3} & \textbf{20} & 0.42 & 0.16 & 0.46 & 0.78 & -0.13 & 0.58 \\ \hline
		\textbf{1M} & \textbf{5} & \textbf{20} & 0.41 & 0.15 & 0.45 & 0.76 & -0.10 & 0.57 \\ \hline
		\textbf{1M} & \textbf{7} & \textbf{20} & 0.40 & 0.14 & 0.44 & 0.77 & -0.09 & 0.56 \\ \hline
	\end{tabular}
\end{table}

\begin{table}[h]
	\centering
	\caption{Risultati di KMeans con leftSkipgram senza costrizioni nella configurazione Random Walk in Illinois}
	\label{leftskipgramrwnckmeansillinois}
	\begin{tabular}{|l|l|l|l|l|l|l|l|l|l|}
		\hline
		\textbf{DB size} & \textbf{Window} & \textbf{RW len.} & \textbf{AMI} & \textbf{ARI} & \textbf{Com} & \textbf{Hom} & \textbf{Silh} & \textbf{V-M} \\ \hline
		\textbf{100K} & \textbf{2} & \textbf{10} & 0.34 & 0.10 & 0.37 & 0.43 & -0.03 & 0.40 \\ \hline
		\textbf{100K} & \textbf{3} & \textbf{10} & 0.37 & 0.20 & 0.40 & 0.45 & -0.03 & 0.42 \\ \hline
		\textbf{100K} & \textbf{5} & \textbf{10} & 0.41 & 0.24 & 0.43 & 0.47 & -0.05 & 0.45 \\ \hline
		\textbf{100K} & \textbf{7} & \textbf{10} & 0.43 & 0.27 & 0.46 & 0.46 & -0.06 & 0.46 \\ \hline
		\textbf{100K} & \textbf{2} & \textbf{15} & 0.34 & 0.14 & 0.36 & 0.46 & \textbf{-0.02} & 0.41 \\ \hline
		\textbf{100K} & \textbf{3} & \textbf{15} & 0.44 & 0.30 & 0.46 & 0.53 & \textbf{-0.02} & 0.49 \\ \hline
		\textbf{100K} & \textbf{5} & \textbf{15} & 0.48 & 0.41 & 0.50 & 0.55 & -0.14 & 0.52 \\ \hline
		\textbf{100K} & \textbf{7} & \textbf{15} & 0.54 & 0.50 & 0.56 & 0.60 & -0.07 & 0.58 \\ \hline
		\textbf{100K} & \textbf{2} & \textbf{20} & 0.34 & 0.15 & 0.36 & 0.48 & -0.02 & 0.42 \\ \hline
		\textbf{100K} & \textbf{3} & \textbf{20} & 0.51 & 0.41 & 0.53 & 0.65 & -0.04 & 0.58 \\ \hline
		\textbf{100K} & \textbf{5} & \textbf{20} & 0.55 & 0.41 & 0.56 & 0.68 & -0.07 & 0.62 \\ \hline
		\textbf{100K} & \textbf{7} & \textbf{20} & 0.55 & 0.51 & 0.57 & 0.62 & -0.09 & 0.59 \\ \hline
		\textbf{500K} & \textbf{2} & \textbf{10} & 0.43 & 0.24 & 0.45 & 0.59 & \textbf{-0.02} & 0.51 \\ \hline
		\textbf{500K} & \textbf{3} & \textbf{10} & 0.53 & 0.39 & 0.55 & 0.67 & -0.04 & 0.60 \\ \hline
		\textbf{500K} & \textbf{5} & \textbf{10} & 0.53 & 0.35 & 0.55 & 0.72 & -0.05 & 0.62 \\ \hline
		\textbf{500K} & \textbf{7} & \textbf{10} & 0.47 & 0.28 & 0.49 & 0.63 & -0.03 & 0.55 \\ \hline
		\textbf{500K} & \textbf{2} & \textbf{15} & 0.52 & 0.32 & 0.54 & 0.71 & \textbf{-0.02} & 0.61 \\ \hline
		\textbf{500K} & \textbf{3} & \textbf{15} & 0.57 & 0.40 & 0.58 & 0.77 & -0.05 & 0.66 \\ \hline
		\textbf{500K} & \textbf{5} & \textbf{15} & 0.58 & 0.39 & 0.59 & 0.79 & -0.05 & 0.68 \\ \hline
		\textbf{500K} & \textbf{7} & \textbf{15} & 0.52 & 0.35 & 0.53 & 0.69 & -0.05 & 0.60 \\ \hline
		\textbf{500K} & \textbf{2} & \textbf{20} & 0.55 & 0.37 & 0.57 & 0.73 & \textbf{-0.02} & 0.64 \\ \hline
		\textbf{500K} & \textbf{3} & \textbf{20} & 0.58 & 0.40 & 0.59 & 0.78 & -0.05 & 0.67 \\ \hline
		\textbf{500K} & \textbf{5} & \textbf{20} & 0.55 & 0.36 & 0.56 & 0.75 & -0.04 & 0.64 \\ \hline
		\textbf{500K} & \textbf{7} & \textbf{20} & 0.51 & 0.32 & 0.53 & 0.71 & -0.04 & 0.61 \\ \hline
		\textbf{1M} & \textbf{2} & \textbf{10} & 0.59 & 0.40 & 0.60 & 0.75 & -0.03 & 0.67 \\ \hline
		\textbf{1M} & \textbf{3} & \textbf{10} & 0.62 & \textbf{0.53} & 0.64 & 0.80 & -0.06 & 0.71 \\ \hline
		\textbf{1M} & \textbf{5} & \textbf{10} & 0.61 & 0.44 & 0.62 & 0.82 & -0.06 & 0.71 \\ \hline
		\textbf{1M} & \textbf{7} & \textbf{10} & 0.60 & 0.43 & 0.61 & 0.81 & -0.05 & 0.69 \\ \hline
		\textbf{1M} & \textbf{2} & \textbf{15} & 0.58 & 0.36 & 0.59 & 0.77 & \textbf{-0.02} & 0.67 \\ \hline
		\textbf{1M} & \textbf{3} & \textbf{15} & \textbf{0.64} & 0.47 & \textbf{0.65} & \textbf{0.85} & -0.04 & \textbf{0.73} \\ \hline
		\textbf{1M} & \textbf{5} & \textbf{15} & 0.53 & 0.33 & 0.55 & 0.73 & -0.05 & 0.62 \\ \hline
		\textbf{1M} & \textbf{7} & \textbf{15} & 0.50 & 0.30 & 0.52 & 0.70 & -0.04 & 0.59 \\ \hline
		\textbf{1M} & \textbf{2} & \textbf{20} & 0.56 & 0.33 & 0.57 & 0.76 & \textbf{-0.02} & 0.65 \\ \hline
		\textbf{1M} & \textbf{3} & \textbf{20} & 0.54 & 0.33 & 0.56 & 0.75 & -0.05 & 0.64 \\ \hline
		\textbf{1M} & \textbf{5} & \textbf{20} & 0.51 & 0.29 & 0.53 & 0.71 & -0.04 & 0.61 \\ \hline
		\textbf{1M} & \textbf{7} & \textbf{20} & 0.51 & 0.30 & 0.53 & 0.71 & -0.04 & 0.60 \\ \hline
	\end{tabular}
\end{table}

\begin{table}[h]
\centering
\caption{Risultati di HDBScan con leftSkipgram senza costrizioni nella configurazione Random Walk in Illinois}
\label{leftskipgramrwnchdbscanillinois}
	\begin{tabular}{|l|l|l|l|l|l|l|l|l|l|}
		\hline
		\textbf{DB size} & \textbf{Window} & \textbf{RW len.} & \textbf{AMI} & \textbf{ARI} & \textbf{Com} & \textbf{Hom} & \textbf{Silh} & \textbf{V-M} \\ \hline
		\textbf{100K} & \textbf{2} & \textbf{10} & 0.11 & 0.16 & 0.33 & 0.12 & \textbf{-0.02} & 0.18 \\ \hline
		\textbf{100K} & \textbf{3} & \textbf{10} & 0.06 & 0.06 & 0.19 & 0.07 & -0.04 & 0.10 \\ \hline
		\textbf{100K} & \textbf{5} & \textbf{10} & 0.10 & 0.05 & 0.23 & 0.11 & -0.11 & 0.15 \\ \hline
		\textbf{100K} & \textbf{7} & \textbf{10} & 0.15 & 0.15 & 0.35 & 0.16 & -0.14 & 0.22 \\ \hline
		\textbf{100K} & \textbf{2} & \textbf{15} & 0.16 & 0.18 & 0.42 & 0.17 & -0.03 & 0.24 \\ \hline
		\textbf{100K} & \textbf{3} & \textbf{15} & 0.20 & 0.13 & 0.39 & 0.23 & -0.08 & 0.29 \\ \hline
		\textbf{100K} & \textbf{5} & \textbf{15} & 0.18 & 0.12 & 0.36 & 0.20 & \textbf{-0.02} & 0.25 \\ \hline
		\textbf{100K} & \textbf{7} & \textbf{15} & 0.19 & 0.20 & 0.41 & 0.21 & -0.03 & 0.27 \\ \hline
		\textbf{100K} & \textbf{2} & \textbf{20} & 0.23 & 0.24 & 0.53 & 0.24 & \textbf{-0.02} & 0.33 \\ \hline
		\textbf{100K} & \textbf{3} & \textbf{20} & 0.18 & 0.11 & 0.38 & 0.20 & -0.15 & 0.26 \\ \hline
		\textbf{100K} & \textbf{5} & \textbf{20} & 0.21 & 0.21 & 0.42 & 0.23 & -0.14 & 0.29 \\ \hline
		\textbf{100K} & \textbf{7} & \textbf{20} & 0.23 & 0.27 & 0.46 & 0.24 & -0.12 & 0.31 \\ \hline
		\textbf{500K} & \textbf{2} & \textbf{10} & 0.16 & 0.20 & 0.46 & 0.17 & \textbf{-0.02} & 0.25 \\ \hline
		\textbf{500K} & \textbf{3} & \textbf{10} & 0.19 & 0.08 & 0.35 & 0.20 & -0.10 & 0.26 \\ \hline
		\textbf{500K} & \textbf{5} & \textbf{10} & 0.16 & 0.05 & 0.37 & 0.18 & -0.10 & 0.24 \\ \hline
		\textbf{500K} & \textbf{7} & \textbf{10} & 0.16 & 0.07 & 0.38 & 0.18 & -0.11 & 0.24 \\ \hline
		\textbf{500K} & \textbf{2} & \textbf{15} & 0.30 & 0.33 & 0.60 & 0.31 & -0.01 & 0.41 \\ \hline
		\textbf{500K} & \textbf{3} & \textbf{15} & 0.20 & 0.10 & 0.39 & 0.22 & \textbf{-0.02} & 0.28 \\ \hline
		\textbf{500K} & \textbf{5} & \textbf{15} & 0.17 & 0.08 & 0.41 & 0.18 & -0.04 & 0.25 \\ \hline
		\textbf{500K} & \textbf{7} & \textbf{15} & 0.19 & 0.14 & 0.42 & 0.21 & \textbf{-0.02} & 0.28 \\ \hline
		\textbf{500K} & \textbf{2} & \textbf{20} & 0.34 & 0.31 & 0.70 & 0.36 & \textbf{-0.02} & 0.47 \\ \hline
		\textbf{500K} & \textbf{3} & \textbf{20} & 0.53 & 0.29 & 0.64 & 0.55 & -0.06 & 0.59 \\ \hline
		\textbf{500K} & \textbf{5} & \textbf{20} & 0.48 & 0.36 & 0.63 & 0.49 & -0.04 & 0.55 \\ \hline
		\textbf{500K} & \textbf{7} & \textbf{20} & 0.53 & 0.47 & 0.64 & 0.54 & -0.04 & 0.59 \\ \hline
		\textbf{1M} & \textbf{2} & \textbf{10} & 0.48 & 0.39 & 0.69 & 0.50 & -0.05 & 0.58 \\ \hline
		\textbf{1M} & \textbf{3} & \textbf{10} & 0.50 & 0.29 & 0.53 & 0.52 & -0.12 & 0.53 \\ \hline
		\textbf{1M} & \textbf{5} & \textbf{10} & 0.55 & 0.44 & 0.57 & 0.74 & -0.11 & 0.65 \\ \hline
		\textbf{1M} & \textbf{7} & \textbf{10} & 0.54 & 0.44 & 0.57 & 0.67 & -0.11 & 0.61 \\ \hline
		\textbf{1M} & \textbf{2} & \textbf{15} & 0.49 & 0.41 & 0.71 & 0.51 & -0.05 & 0.60 \\ \hline
		\textbf{1M} & \textbf{3} & \textbf{15} & 0.54 & 0.31 & 0.56 & 0.72 & -0.09 & 0.63 \\ \hline
		\textbf{1M} & \textbf{5} & \textbf{15} & 0.49 & 0.31 & 0.52 & \textbf{0.78} & -0.10 & 0.62 \\ \hline
		\textbf{1M} & \textbf{7} & \textbf{15} & 0.45 & 0.20 & 0.48 & 0.75 & -0.10 & 0.58 \\ \hline
		\textbf{1M} & \textbf{2} & \textbf{20} & \textbf{0.60} & \textbf{0.52} & \textbf{0.76} & 0.61 & -0.03 & \textbf{0.68} \\ \hline
		\textbf{1M} & \textbf{3} & \textbf{20} & 0.47 & 0.25 & 0.51 & 0.76 & -0.11 & 0.61 \\ \hline
		\textbf{1M} & \textbf{5} & \textbf{20} & 0.45 & 0.21 & 0.49 & 0.74 & -0.10 & 0.59 \\ \hline
		\textbf{1M} & \textbf{7} & \textbf{20} & 0.42 & 0.17 & 0.46 & 0.74 & -0.10 & 0.57 \\ \hline
	\end{tabular}
\end{table}

\begin{table}[h]
\centering
\caption{Risultati di KMeans con normalSkipgram senza costrizioni nella configurazione Random Walk in Illinois}
\label{normalskipgramrwnckmeansillinois}
	\begin{tabular}{|l|l|l|l|l|l|l|l|l|l|}
		\hline
		\textbf{DB size} & \textbf{Window} & \textbf{RW len.} & \textbf{AMI} & \textbf{ARI} & \textbf{Com} & \textbf{Hom} & \textbf{Silh} & \textbf{V-M} \\ \hline
		\textbf{100K} & \textbf{2} & \textbf{10} & 0.65 & 0.50 & 0.66 & 0.83 & -0.05 & 0.74 \\ \hline
		\textbf{100K} & \textbf{3} & \textbf{10} & 0.65 & 0.57 & 0.67 & 0.81 & -0.05 & 0.73 \\ \hline
		\textbf{100K} & \textbf{5} & \textbf{10} & 0.66 & 0.55 & 0.68 & 0.82 & -0.05 & 0.74 \\ \hline
		\textbf{100K} & \textbf{7} & \textbf{10} & 0.59 & 0.49 & 0.60 & 0.75 & -0.05 & 0.67 \\ \hline
		\textbf{100K} & \textbf{2} & \textbf{15} & 0.64 & 0.46 & 0.66 & 0.83 & -0.04 & 0.73 \\ \hline
		\textbf{100K} & \textbf{3} & \textbf{15} & 0.60 & 0.38 & 0.61 & 0.79 & -0.04 & 0.69 \\ \hline
		\textbf{100K} & \textbf{5} & \textbf{15} & 0.60 & 0.37 & 0.61 & 0.79 & -0.05 & 0.69 \\ \hline
		\textbf{100K} & \textbf{7} & \textbf{15} & 0.60 & 0.42 & 0.62 & 0.77 & -0.05 & 0.68 \\ \hline
		\textbf{100K} & \textbf{2} & \textbf{20} & \textbf{0.69} & \textbf{0.61} & \textbf{0.70} & 0.87 & -0.04 & \textbf{0.78} \\ \hline
		\textbf{100K} & \textbf{3} & \textbf{20} & 0.61 & 0.45 & 0.62 & 0.78 & -0.04 & 0.69 \\ \hline
		\textbf{100K} & \textbf{5} & \textbf{20} & 0.58 & 0.41 & 0.60 & 0.77 & -0.04 & 0.67 \\ \hline
		\textbf{100K} & \textbf{7} & \textbf{20} & 0.62 & 0.47 & 0.63 & 0.81 & -0.03 & 0.71 \\ \hline
		\textbf{500K} & \textbf{2} & \textbf{10} & 0.67 & 0.47 & 0.69 & \textbf{0.90} & -0.03 & \textbf{0.78} \\ \hline
		\textbf{500K} & \textbf{3} & \textbf{10} & 0.62 & 0.38 & 0.63 & 0.84 & -0.03 & 0.72 \\ \hline
		\textbf{500K} & \textbf{5} & \textbf{10} & 0.66 & 0.46 & 0.67 & 0.88 & \textbf{-0.02} & 0.76 \\ \hline
		\textbf{500K} & \textbf{7} & \textbf{10} & 0.58 & 0.31 & 0.60 & 0.75 & \textbf{-0.02} & 0.66 \\ \hline
		\textbf{500K} & \textbf{2} & \textbf{15} & 0.65 & 0.44 & 0.66 & 0.89 & -0.03 & 0.76 \\ \hline
		\textbf{500K} & \textbf{3} & \textbf{15} & 0.66 & 0.45 & 0.67 & 0.89 & -0.03 & 0.77 \\ \hline
		\textbf{500K} & \textbf{5} & \textbf{15} & 0.60 & 0.36 & 0.61 & 0.83 & \textbf{-0.02} & 0.70 \\ \hline
		\textbf{500K} & \textbf{7} & \textbf{15} & 0.56 & 0.29 & 0.58 & 0.76 & \textbf{-0.02} & 0.66 \\ \hline
		\textbf{500K} & \textbf{2} & \textbf{20} & 0.61 & 0.38 & 0.63 & 0.84 & -0.04 & 0.72 \\ \hline
		\textbf{500K} & \textbf{3} & \textbf{20} & 0.64 & 0.42 & 0.65 & 0.87 & -0.03 & 0.74 \\ \hline
		\textbf{500K} & \textbf{5} & \textbf{20} & 0.58 & 0.36 & 0.59 & 0.79 & -0.03 & 0.68 \\ \hline
		\textbf{500K} & \textbf{7} & \textbf{20} & 0.58 & 0.36 & 0.60 & 0.81 & \textbf{-0.02} & 0.69 \\ \hline
		\textbf{1M} & \textbf{2} & \textbf{10} & 0.62 & 0.37 & 0.63 & 0.85 & -0.04 & 0.73 \\ \hline
		\textbf{1M} & \textbf{3} & \textbf{10} & 0.61 & 0.36 & 0.62 & 0.83 & -0.03 & 0.71 \\ \hline
		\textbf{1M} & \textbf{5} & \textbf{10} & 0.64 & 0.43 & 0.65 & 0.88 & -0.03 & 0.75 \\ \hline
		\textbf{1M} & \textbf{7} & \textbf{10} & 0.62 & 0.38 & 0.63 & 0.85 & -0.03 & 0.73 \\ \hline
		\textbf{1M} & \textbf{2} & \textbf{15} & 0.63 & 0.42 & 0.64 & 0.87 & -0.04 & 0.74 \\ \hline
		\textbf{1M} & \textbf{3} & \textbf{15} & 0.61 & 0.36 & 0.62 & 0.84 & -0.04 & 0.71 \\ \hline
		\textbf{1M} & \textbf{5} & \textbf{15} & 0.59 & 0.33 & 0.60 & 0.81 & -0.03 & 0.69 \\ \hline
		\textbf{1M} & \textbf{7} & \textbf{15} & 0.60 & 0.37 & 0.61 & 0.83 & -0.03 & 0.71 \\ \hline
		\textbf{1M} & \textbf{2} & \textbf{20} & 0.63 & 0.38 & 0.64 & 0.86 & -0.04 & 0.73 \\ \hline
		\textbf{1M} & \textbf{3} & \textbf{20} & 0.59 & 0.37 & 0.60 & 0.82 & -0.03 & 0.70 \\ \hline
		\textbf{1M} & \textbf{5} & \textbf{20} & 0.61 & 0.37 & 0.62 & 0.83 & -0.03 & 0.71 \\ \hline
		\textbf{1M} & \textbf{7} & \textbf{20} & 0.59 & 0.35 & 0.60 & 0.81 & -0.03 & 0.69 \\ \hline
	\end{tabular}
\end{table}

\begin{table}[h]
\centering
\caption{Risultati di HDBScan con normalSkipgram senza costrizioni nella configurazione Random Walk in Illinois}
\label{normalskipgramrwnchdbscanillinois}
	\begin{tabular}{|l|l|l|l|l|l|l|l|l|l|}
		\hline
		\textbf{DB size} & \textbf{Window} & \textbf{RW len.} & \textbf{AMI} & \textbf{ARI} & \textbf{Com} & \textbf{Hom} & \textbf{Silh} & \textbf{V-M} \\ \hline
		\textbf{100K} & \textbf{2} & \textbf{10} & 0.40 & 0.09 & 0.45 & 0.46 & -0.09 & 0.46 \\ \hline
		\textbf{100K} & \textbf{3} & \textbf{10} & 0.32 & 0.02 & 0.40 & 0.37 & -0.10 & 0.39 \\ \hline
		\textbf{100K} & \textbf{5} & \textbf{10} & 0.27 & 0.02 & 0.40 & 0.32 & -0.09 & 0.36 \\ \hline
		\textbf{100K} & \textbf{7} & \textbf{10} & 0.28 & 0.03 & 0.42 & 0.33 & -0.09 & 0.37 \\ \hline
		\textbf{100K} & \textbf{2} & \textbf{15} & \textbf{0.46} & 0.15 & \textbf{0.49} & 0.56 & -0.08 & 0.53 \\ \hline
		\textbf{100K} & \textbf{3} & \textbf{15} & 0.39 & 0.08 & 0.45 & 0.43 & -0.08 & 0.44 \\ \hline
		\textbf{100K} & \textbf{5} & \textbf{15} & 0.34 & 0.08 & 0.45 & 0.38 & -0.07 & 0.42 \\ \hline
		\textbf{100K} & \textbf{7} & \textbf{15} & 0.30 & 0.05 & 0.44 & 0.34 & -0.07 & 0.38 \\ \hline
		\textbf{100K} & \textbf{2} & \textbf{20} & 0.41 & 0.07 & 0.45 & 0.51 & -0.11 & 0.48 \\ \hline
		\textbf{100K} & \textbf{3} & \textbf{20} & 0.36 & 0.02 & 0.41 & 0.43 & -0.12 & 0.42 \\ \hline
		\textbf{100K} & \textbf{5} & \textbf{20} & 0.33 & 0.02 & 0.40 & 0.38 & -0.10 & 0.39 \\ \hline
		\textbf{100K} & \textbf{7} & \textbf{20} & 0.30 & 0.01 & 0.39 & 0.35 & -0.09 & 0.37 \\ \hline
		\textbf{500K} & \textbf{2} & \textbf{10} & 0.42 & 0.16 & 0.46 & 0.69 & -0.09 & 0.55 \\ \hline
		\textbf{500K} & \textbf{3} & \textbf{10} & 0.40 & 0.14 & 0.45 & 0.66 & -0.08 & 0.53 \\ \hline
		\textbf{500K} & \textbf{5} & \textbf{10} & 0.41 & 0.14 & 0.45 & 0.68 & -0.08 & 0.54 \\ \hline
		\textbf{500K} & \textbf{7} & \textbf{10} & 0.39 & 0.08 & 0.43 & 0.58 & \textbf{-0.06} & 0.49 \\ \hline
		\textbf{500K} & \textbf{2} & \textbf{15} & 0.43 & 0.17 & 0.47 & 0.75 & -0.09 & \textbf{0.58} \\ \hline
		\textbf{500K} & \textbf{3} & \textbf{15} & 0.39 & 0.13 & 0.44 & 0.69 & -0.09 & 0.54 \\ \hline
		\textbf{500K} & \textbf{5} & \textbf{15} & 0.41 & 0.12 & 0.45 & 0.66 & -0.07 & 0.54 \\ \hline
		\textbf{500K} & \textbf{7} & \textbf{15} & 0.42 & 0.13 & 0.46 & 0.67 & \textbf{-0.06} & 0.55 \\ \hline
		\textbf{500K} & \textbf{2} & \textbf{20} & 0.41 & \textbf{0.18} & 0.46 & 0.75 & -0.09 & 0.57 \\ \hline
		\textbf{500K} & \textbf{3} & \textbf{20} & 0.39 & 0.12 & 0.44 & 0.70 & -0.08 & 0.54 \\ \hline
		\textbf{500K} & \textbf{5} & \textbf{20} & 0.40 & 0.13 & 0.44 & 0.69 & -0.07 & 0.54 \\ \hline
		\textbf{500K} & \textbf{7} & \textbf{20} & 0.41 & 0.17 & 0.46 & 0.73 & -0.07 & 0.56 \\ \hline
		\textbf{1M} & \textbf{2} & \textbf{10} & 0.39 & 0.15 & 0.44 & 0.77 & -0.11 & 0.56 \\ \hline
		\textbf{1M} & \textbf{3} & \textbf{10} & 0.39 & 0.14 & 0.43 & 0.76 & -0.11 & 0.55 \\ \hline
		\textbf{1M} & \textbf{5} & \textbf{10} & 0.38 & 0.14 & 0.43 & 0.79 & -0.11 & 0.56 \\ \hline
		\textbf{1M} & \textbf{7} & \textbf{10} & 0.38 & 0.13 & 0.43 & 0.75 & -0.10 & 0.55 \\ \hline
		\textbf{1M} & \textbf{2} & \textbf{15} & 0.38 & 0.14 & 0.43 & 0.77 & -0.11 & 0.55 \\ \hline
		\textbf{1M} & \textbf{3} & \textbf{15} & 0.39 & 0.14 & 0.44 & 0.78 & -0.11 & 0.56 \\ \hline
		\textbf{1M} & \textbf{5} & \textbf{15} & 0.39 & 0.14 & 0.44 & 0.78 & -0.10 & 0.56 \\ \hline
		\textbf{1M} & \textbf{7} & \textbf{15} & 0.39 & 0.13 & 0.44 & 0.77 & -0.10 & 0.56 \\ \hline
		\textbf{1M} & \textbf{2} & \textbf{20} & 0.37 & 0.13 & 0.42 & 0.76 & -0.13 & 0.54 \\ \hline
		\textbf{1M} & \textbf{3} & \textbf{20} & 0.39 & 0.14 & 0.44 & 0.77 & -0.12 & 0.56 \\ \hline
		\textbf{1M} & \textbf{5} & \textbf{20} & 0.39 & 0.14 & 0.44 & 0.78 & -0.11 & 0.56 \\ \hline
		\textbf{1M} & \textbf{7} & \textbf{20} & 0.40 & 0.15 & 0.45 & \textbf{0.79} & -0.11 & 0.57 \\ \hline
	\end{tabular}
\end{table}

\begin{table}[h]
\centering
\caption{Risultati migliori tra leftSkipgram, normalSkipgram e LINE in Illinois}
\label{bestleftnormalline}
	\begin{tabular}{|l|l|l|l|l|l|l|l|l|l|}
		\hline
		& \textbf{Clustering} & \textbf{AMI} & \textbf{ARI} & \textbf{Com} & \textbf{Hom} & \textbf{Silh} & \textbf{V-M} \\ \hline
		\textbf{leftSkipgram-lc} & \textbf{KMeans} & 0.62 & 0.42 & 0.63 & 0.83 & \textbf{-0.03} & 0.72 \\ \hline
		\textbf{leftSkipgram-lc} & \textbf{HDBScan} & 0.44 & 0.19 & 0.47 & 0.78 & -0.14 & 0.59 \\ \hline
		\textbf{normalSkipgram-lc} & \textbf{KMeans} & 0.66 & 0.46 & 0.67 & \textbf{0.89} & \textbf{-0.03} & 0.76 \\ \hline
		\textbf{normalSkipgram-lc} & \textbf{HDBScan} & 0.43 & 0.17 & 0.47 & 0.74 & -0.08 & 0.58 \\ \hline
		\textbf{LINE-1-lc} & \textbf{KMeans} & 0.48 & 0.31 & 0.50 & 0.64 & \textbf{-0.03} & 0.56 \\ \hline
		\textbf{LINE-1-lc} & \textbf{HDBScan} & 0.52 & 0.34 & 0.58 & 0.56 & -0.09 & 0.57 \\ \hline
		\textbf{LINE-2-lc} & \textbf{KMeans} & 0.30 & 0.13 & 0.34 & 0.42 & -0.05 & 0.37 \\ \hline
		\textbf{LINE-2-lc} & \textbf{HDBScan} & 0.19 & -0.01 & 0.34 & 0.24 & -0.12 & 0.28 \\ \hline
		\textbf{leftSkipgram-nc} & \textbf{KMeans} & 0.64 & 0.47 & 0.65 & 0.85 & -0.04 & 0.73 \\ \hline
		\textbf{leftSkipgram-nc} & \textbf{HDBScan} & 0.60 & 0.52 & \textbf{0.76} & 0.61 & \textbf{-0.03} & 0.68 \\ \hline
		\textbf{normalSkipgram-nc} & \textbf{KMeans} & \textbf{0.69} & \textbf{0.61} & 0.70 & 0.87 & -0.04 & \textbf{0.78} \\ \hline
		\textbf{normalSkipgram-nc} & \textbf{HDBScan} & 0.46 & 0.15 & 0.49 & 0.56 & -0.08 & 0.53 \\ \hline
		\textbf{LINE-1-nc} & \textbf{KMeans} & 0.51 & 0.29 & 0.52 & 0.66 & -0.04 & 0.58 \\ \hline
		\textbf{LINE-1-nc} & \textbf{HDBScan} & 0.39 & 0.18 & 0.44 & 0.70 & -0.10 & 0.54\\ \hline
		\textbf{LINE-2-nc} & \textbf{KMeans} & 0.48 & 0.29 & 0.50 & 0.63 & -0.05 & 0.55 \\ \hline
		\textbf{LINE-2-nc} & \textbf{HDBScan} & 0.45 & 0.18 & 0.49 & 0.50 & -0.09 & 0.49 \\ \hline
	\end{tabular}
	\begin{tabular}{|l|l|l|l|l|}
		\hline
		& \textbf{Clustering} & \textbf{DB size} & \textbf{Window} & \textbf{RW len.} \\ \hline
		\textbf{leftSkipgram-lc} & \textbf{KMeans} & 500K & 7 & 20 \\ \hline
		\textbf{leftSkipgram-lc} & \textbf{HDBScan} & 1M & 5 & 15 \\ \hline
		\textbf{normalSkipgram-lc} & \textbf{KMeans} & 500K & 5 & 10 \\ \hline
		\textbf{normalSkipgram-lc} & \textbf{HDBScan} & 500K & 7 & 20 \\ \hline
		\textbf{leftSkipgram-nc} & \textbf{KMeans} & 1M & 3 & 15 \\ \hline
		\textbf{leftSkipgram-nc} & \textbf{HDBScan} & 1M & 2 & 20 \\ \hline
		\textbf{normalSkipgram-nc} & \textbf{KMeans} & 100K & 2 & 20 \\ \hline
		\textbf{normalSkipgram-nc} & \textbf{HDBScan} & 100K & 2 & 15 \\ \hline
	\end{tabular}
\end{table}

\subsubsection{Testo}

\begin{table}[h]
\centering
\caption{Risultati di Doc2Vec e TF-IDF in Illinois}
\label{doc2vectfidfillinois}
	\begin{tabular}{|l|l|l|l|l|l|l|l|l|l|}
		\hline
		& \textbf{Clustering} & \textbf{AMI} & \textbf{ARI} & \textbf{Com} & \textbf{Hom} & \textbf{Silh} & \textbf{V-M} \\ \hline
		\textbf{Doc2Vec} & \textbf{KMeans} & 0.53 & 0.38 & 0.55 & 0.71 & -0.11 & 0.62 \\ \hline
		\textbf{Doc2Vec} & \textbf{HDBScan} & \textbf{0.72} & \textbf{0.80} & \textbf{0.75} & 0.73 & -0.02 & \textbf{0.74} \\ \hline
		\textbf{TF-IDF} & \textbf{KMeans} & 0.60 & 0.38 & 0.62 & \textbf{0.84} & \textbf{0.00} & 0.71\\ \hline
		\textbf{TF-IDF} & \textbf{HDBScan} & 0.56 & 0.41 & 0.59 & 0.68 & -0.04 & 0.63 \\ \hline
	\end{tabular}
\end{table}

\subsubsection{Combinato}

\begin{table}[h]
\centering
\caption{Risultati di Combinato in Illinois}
\label{combillinois}
	\begin{tabular}{|l|l|l|l|l|l|l|l|l|l|}
		\hline
		& \textbf{Clustering} & \textbf{AMI} & \textbf{ARI} & \textbf{Com} & \textbf{Hom} & \textbf{Silh} & \textbf{V-M} \\ \hline
		\textbf{left-lc + Doc2Vec} & \textbf{KMeans} & 0.60 & 0.41 & 0.61 & 0.81 & -0.03 & 0.70 \\ \hline
		\textbf{left-lc + Doc2Vec} & \textbf{HDBScan} & 0.43 & 0.19 & 0.47 & 0.77 & -0.15 & 0.58 \\ \hline
		\textbf{left-lc + TF-IDF} & \textbf{KMeans} & 0.69 & 0.49 & 0.70 & \textbf{0.92} & -0.03 & 0.80 \\ \hline
		\textbf{left-lc + TF-IDF} & \textbf{HDBScan} & 0.76 & 0.72 & 0.77 & 0.82 & -0.02 & 0.79 \\ \hline
		\textbf{normal-lc + Doc2Vec} & \textbf{KMeans} & 0.66 & 0.47 & 0.67 & 0.88 & -0.03 & 0.76 \\ \hline
		\textbf{normal-lc + Doc2Vec} & \textbf{HDBScan} & 0.43 & 0.19 & 0.47 & 0.73 & -0.08 & 0.57 \\ \hline
		\textbf{normal-lc + TF-IDF} & \textbf{KMeans} & 0.69 & 0.49 & 0.70 & \textbf{0.92} & -0.02 & 0.79 \\ \hline
		\textbf{normal-lc + TF-IDF} & \textbf{HDBScan} & 0.55 & 0.36 & 0.58 & 0.77 & -0.04 & 0.66 \\ \hline
		\textbf{left-nc + Doc2Vec} & \textbf{KMeans} & 0.67 & 0.48 & 0.68 & 0.88 & -0.04 & 0.77 \\ \hline
		\textbf{left-nc + Doc2Vec} & \textbf{HDBScan} & 0.61 & 0.53 & 0.78 & 0.62 & -0.03 & 0.69 \\ \hline
		\textbf{left-nc + TF-IDF} & \textbf{KMeans} & 0.67 & 0.54 & 0.68 & 0.87 & -0.03 & 0.76 \\ \hline
		\textbf{left-nc + TF-IDF} & \textbf{HDBScan} & \textbf{0.78} & \textbf{0.76} & \textbf{0.83} & 0.79 & -0.03 & \textbf{0.81} \\ \hline
		\textbf{normal-nc + Doc2Vec} & \textbf{KMeans} & 0.65 & 0.47 & 0.66 & 0.88 & -0.04 & 0.76 \\ \hline
		\textbf{normal-nc + Doc2Vec} & \textbf{HDBScan} & 0.45 & 0.17 & 0.48 & 0.64 & -0.10 & 0.55 \\ \hline
		\textbf{normal-nc + TF-IDF} & \textbf{KMeans} & 0.68 & 0.47 & 0.69 & \textbf{0.92} & -0.03 & 0.79 \\ \hline
		\textbf{normal-nc + TF-IDF} & \textbf{HDBScan} & 0.77 & 0.68 & 0.78 & 0.83 & \textbf{-0.01} & 0.80 \\ \hline
	\end{tabular}
\end{table}

\subsection{cs.ox.ac.uk}

\subsection{cs.priceton.edu}

\subsection{cs.stanford.edu}

\bibliographystyle{plain}
\bibliography{./../Bibliografia}                % database di biblatex 

\end{document}

\chapter{Conclusioni e sviluppi futuri}
\label{cap:capitolo5}
In questa tesi si � trattato del Clustering di pagine Web, proponendo un nuovo metodo che combina l'informazione estratta dal contenuto dei testi delle pagine e quella dalla struttura ad hyperlink del sito Web in un singolo spazio vettoriale, che pu� essere usato dagli algoritmi di Clustering tradizionali meglio performanti. Durante la sperimentazione si � cercato di capire se utilizzare Skip-Gram che considera solo il contesto sinistro potesse migliorare la qualit� dei raggruppamenti prodotti dai vari algoritmi di Clustering, se effettivamente combinare l'informazione del contenuto e della struttura potesse aumentare le performance del processo di raggruppamento e se utilizzare le Liste Web per ridurre il rumore potesse migliorare i risultati del Clustering.
\\
I risultati della sperimentazione ci mostrano che il testo delle pagine e la struttura del sito Web forniscono informazioni diverse e complementari che possono migliorare l'efficacia degli algoritmi di Clustering. Non sono state riscontrate differenze statisticamente significative nell'utilizzo delle liste di costrizione e nell'applicazione di Skip-Gram modificato.
\\
Futuri lavori potrebbero incentrarsi sull'applicazione della metodologia descritta in questa tesi su pi� siti Web e meno strutturati, in modo da osservare se l'uso delle Liste influenza il processo di Clustering delle pagine Web.

\bibliographystyle{plain}
\bibliography{./Bibliografia}                % database di biblatex

\end{document}