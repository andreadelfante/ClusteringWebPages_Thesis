\documentclass[10pt]{article}
\usepackage[usenames]{color} %usato per il colore
\usepackage{amssymb} %maths
\usepackage{amsmath} %maths
\usepackage[utf8]{inputenc} %utile per scrivere direttamente in caratteri accentuati
\begin{document}
\begin{align*}\section{Sitemap (da omonimo paper)}
Uno dei problemi riconosciuti nel Web Mining consiste nella costruzione automatica delle gerarchie di pagine web, tipicamente chiamate Sitemap. Una sitemap (o mappa del sito, in italiano) è la rappresentazione esplicita della progettazione del sito, codificata dai progettisti di User Experiente e dagli architetti dell'informazione, organizzata in maniera gerarchica: un utente inizia a navigare all'interno del sito partendo dalla homepage o da una pagina trovata da un motore di ricerca e, attraverso l'uso di sistemi di navigazione forniti dal sito web, può esplorare le varie informazioni contenute e trovare quella desiderata.
La costruzione di una sitemap non è un processo semplice, specialmente per siti web che hanno una grande quantità di contenuti e con gerarchie logicamente profondi e vaste.

%vedi che altro riesci a trovare per collegare questi due concetti

Una pagina web è caratterizzata da rappresentazioni multiple, come quella testuale (mediante i termini e le parole che si trovano nella pagina web), visuale (composta dall'informazione del rendering del sito) e strutturale (formata dai tag HTML). 

\paragraph{Sito web} Un sito web è un grafo diretto G = (V, E), dove V è l'insieme delle pagine web ed E è l'insieme degli hyperlinks. nella maggior parte dei casi, la homepage h di un sito web rappresenta la pagina di entrata di quel sito, permettendo al sito di essere visto come un grafo diretto radicato (ovvero avente una radice).

\paragraph{Rappresentazione strutturale} Una pagina web è caratterizzata da una rappresentazione strutturale, composta da elementi inscritti in tag HTML e organizzati secondo una struttura ad albero. Tali tag possono essere applicato a porzioni di testo, hyperlinks e dati multimediali per dare loro un significato differente e una diversa renderizzazione della pagina web.

\paragraph{Rappresentazione visuale} Quando una pagina web viene interpretata dal browser, si dice che questa viene renderizzata dal browser: gli elementi che sono contenuti nella pagina web vengono rappresentati da scatole rettangolari, le quali possono essere affiancate oppure innestate, creando un albero chiamato Rendered Box Tree. Utilizzando un sistema di coordinate avente origine nell'angolo in alto a sinistra, tutte le posizioni degli elementi della pagina web vengono determinati. Il Rendered Box Tree, inoltre, può essere generato da qualsiasi browser che segue le specifiche di rendering del W3C.

\paragraph{Web List} Una Web List è una collezione di due o più elementi web (chiamati record di dati) codificati come delle Rendered Box che hanno una struttura HTML simile, e sono allineati o adiacenti visualmente.

\paragraph{Sequenze} Sia G = (V, E) un sito web: una sequenza S è definita come $S = <t_1, t_2, ..., t_n>$, dove ogni elemento $t_j \in V $ denota che la pagina web al passo j-esimo nel percorso di navigazione S. Una sequenza $S = <t_1, t_2, ..., t_n>$ è una sottosequenza di una sequenza $S' = <a_1, a_2, ..., a_n>$\end{align*}
\end{document}