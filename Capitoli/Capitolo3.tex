\documentclass[a4paper,12pt,oneside]{book}
\usepackage[usenames]{color} %usato per il colore
\usepackage{amssymb} %maths
\usepackage{amsmath} %maths
\usepackage[latin1]{inputenc}
\usepackage[pdftex]{graphicx} 
\usepackage{filecontents}
\usepackage[italian]{babel}
\usepackage{setspace}

\graphicspath{ {./../Immagini/} }

\begin{document}

\addtolength{\oddsidemargin}{+1,0cm} 
\addtolength{\evensidemargin}{+1,0cm} 
\onehalfspacing

\section{Web Crawling}

\section{Costruzione del dataset}

\subsection{Random Walks}

\subsection{Generazione delle sequenze}

\subsection{Esempio di dataset}
In questa sezione si spiega come � stato strutturato il dataset per la sperimentazione. 
I file sono stati generati dal Web Crawling e dai Random Walks, e che sono usati per effettuare la sperimentazione descritta in questa tesi [...][aggiungi riferimento].\\
Per spiegare il dataset � stato preso come esempio il sito del dipartimento di informatica di Stanford, CA: \texttt{cs.stanford.edu}.

\paragraph{edges.txt}
Contiene gli archi tra i nodi del grafo del sito web, i quali non sono altro che gli hyperlinks tra le pagine. Questo file viene usato per la generazione delle sequenze dall'algoritmo dei Random Walks.
\\\\
\texttt{
3	69\\
3	101\\
3	88\\
3	115\\
3	5\\
3	120\\
3	10\\
3	56\\
...
}

\paragrath{embeddings_with_b.txt, embeddings_no_b.txt, embeddings_normal.txt, embeddings_line_first.txt, embeddings_line_second.txt}
Questi file contengono gli embeddings, i.e. lo spazio dei vettori degli url appresi, 

\section{Web page clustering}

%aggiungi visualizzazione dati

%\bibliographystyle{plain}
%\bibliography{./../Bibliografia}                % database di biblatex 

\end{document}