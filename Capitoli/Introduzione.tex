\cleardoublepage
\chapter*{Introduzione}

L'Informatica � diventata uno dei pilastri fondamentali su cui si regge la societ� moderna, ovvero quella dell'informazione. Ogni giorno vengono prodotti un gran numero di dati liberamente fruibili da chiunque, che vengono archiviati e organizzati in maniera automatica ed economica. Produrre nuova conoscenza significa eseguire un processo di elaborazione dei dati che possa arricchire il sapere pregresso. Aumentare il bagaglio della conoscenza significa, quindi, facilitare i processi decisionali.
\\\\
Questo problema � particolarmente sentito nel World Wide Web. Il Web � il pi� grande, eterogeneo e dinamico contenitore di sapere liberamente fruibile da chiunque. Queste caratteristiche, per�, rendono il processo di elaborazione dei dati e di produzione di nuova conoscenza una sfida impegnativa.
\\\\
Sorge, quindi, una nuova problematica: accedere all'enorme mole di dati archiviati nel Web in maniera veloce e mirata. Una possibile soluzione consiste nell'utilizzo di tecniche di Clustering su pagine Web, ossia assegnarle a gruppi in cui si trovano elementi appartenenti alla stessa tipologia semantica (per esempio pagine di processori, corsi e prodotti).
\\
Il Clustering di pagine Web non si propone come una nuova metodologia: in letteratura, infatti, esistono algoritmi che sfruttano o la struttura organizzativa di un sito Web o il testo contenuto nelle pagine, considerandole come documenti indipendenti tra loro.
\\\\
Ma nel contesto del Web, le pagine non possono essere trattate come documenti a se stanti, bisognerebbe piuttosto cercare di utilizzare l'informazione codificata nella struttura ad hyperlink del sito.
\\
La metodologia descritta in questa tesi non considera, infatti, un sito Web come una collezione di documenti testuali indipendenti tra loro, ma cerca di combinare informazioni relative al contenuto con quelle strutturali, in modo che due pagine Web vengano considerate simili se caratterizzate da una simile distribuzione di termini e abbiano una relazione di tipo diretta o indiretta, diretta se un hyperlink porta immediatamente ad una pagina, indiretta se vi sono pagine intermedie tra quella di partenza e di destinazione.
\\\\
Si definisce di seguito la struttura di questo lavoro di tesi.\\
Nel capitolo 1 ci si occuper� di descrivere i concetti essenziali per comprendere a pieno il campo applicativo in cui verr� effettuata la sperimentazione.
Nel capitolo 2 si parler� della frontiera attuale dell'Informatica in questo campo, descrivendo i risultati dei lavori correlati a quello descritto in questa tesi.
Nel capitolo 3 verranno descritti i passi eseguiti per questa sperimentazione, spiegandoli concettualmente e riportando gli algoritmi in pseudocodice.
Nel capitolo 4 si descriver� la sperimentazione effettuata, spiegando le metriche utilizzate, le configurazioni utilizzate e riportando le tabelle con i risultati utili per spiegare i valori ottenuti.
Nel capitolo 5, infine, si parler� dei possibili sviluppi futuri di questa attivit� di ricerca.