In questa tesi si � trattato del Clustering di pagine Web, proponendo un nuovo metodo che combina l'informazione estratta dal contenuto dei testi delle pagine e quella dalla struttura ad hyperlink del sito Web in un singolo spazio vettoriale, che pu� essere usato dagli algoritmi di Clustering tradizionali meglio performanti. Durante la sperimentazione, si � cercato di capire se l'utilizzo di Skip-Gram che considera solo il contesto sinistro potesse migliorare la qualit� dei raggruppamenti prodotti dai vari algoritmi di Clustering, se effettivamente combinare l'informazione del contenuto e della struttura potesse aumentare le performance del processo di raggruppamento e se utilizzare le Liste Web per ridurre il rumore potesse migliorare i risultati del Clustering.
\\
I risultati della sperimentazione ci mostrano che il testo delle pagine e la struttura del sito Web forniscono informazioni diverse e complementari che possono migliorare l'efficacia degli algoritmi di Clustering. Non sono state riscontrate differenze statisticamente significative nell'utilizzo delle liste di costrizione e nell'applicazione di Skip-Gram modificato.
\\
Futuri lavori potrebbero incentrarsi sull'applicazione della metodologia descritta in questa tesi su pi� siti Web e meno strutturati, in modo da osservare se l'uso delle Liste influenza il processo di Clustering delle pagine Web.